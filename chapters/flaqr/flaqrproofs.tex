\if 0
\begin{definition}[Plug] plugging of hole to evaluation context $E$\\
\ensuremath{
1.[~][e] = e \\
2.(E e')[e] = (E [e]) e' \\
3.(w E')[e] = w (E'[e]) \\
4.(E \tau)[e] = (E[e])\tau \\
5.(\pair{E}{e})[e] = \pair{E[e]}{e} \\
6.(\pair{w}{E})[e] = \pair{w}{E[e]} \\
7.(\return{\ell }{E} )[e] = \return{\ell }{E[e]} \\
8.(\proji{E})[e] = \proji{E[e]} \\
9.(\inji{\sumtype{\tau_1}{\tau_2}}{E})[e] = \inji{\sumtype{\tau_1}{\tau_2}}{E[e]} \\
10.(\bind{x}{E}{e'})[e] = \bind{x}{E[e]}{e'}\\ 
11.(\casexp{E}{x}{e_1}{e_2}{\sumtype{\tau_1}{\tau_2}})[e] \\= \casexp{E[e]}{x}{e_1}{e_2}{\sumtype{\tau_1}{\tau_2}}\\
14.(\ret{E}{p})[e] = \ret{E[e]}{p}\\
15. (\select{E}{e'})[e] = \select{E[e]}{e'} \\
16. (\select{w}{E})[e] = \select{w}{E[e]}  \\
%17. (\compEnd{\tau}{E}{e'})[e] = \compEnd{\tau}{E[e]}{e'}\\
%18. (\compEnd{\tau}{w}{E})[e] = \compEnd{\tau}{w}{E[e]} \\ 
}
\end{definition}
\fi 
%\begin{lemma}[RunT]
%If $\runnable{\Pi}{\tau}{c}$ then $\forall{e}.$
%\TValGpcw{e}{\tau}
%\end{lemma}

\begin{lemma}[UniqueType]
If~{\TValGpcw{e}{\tau}}~ and~ {\TValGput{e}{\mathring{\tau}}}~then~
$\tau = \tau'$
\end{lemma}
\textbf{Proof.}
Proof by induction on typing derivation of $e$. \\
\textbf{Case VAR.}
Given, 
\begin{align}
{\TValGpcw{x}{\tau}}  \label{vu1} \\
{\TValGput{x}{\mathring{\tau}}} \label{vu1.1} \\
\intertext{inverting \ref{vu1} we get}
\Gamma(x)=\tau  \label{vu2} \\ 
\rafjudge{\Pi}{c}{pc} \label{vu3} \\
\intertext{inverting \ref{vu1.1} we get}
\Gamma(x)=\mathring{\tau}  \label{vu4} \\ 
\rafjudge{Π}{c}{\mathring{pc}} \label{vu5} \\
\intertext{$\Gamma$ in \ref{vu1} and \ref{vu1.1} are same, so from 
\ref{vu2} and \ref{vu4} we get}
\tau = \mathring{\tau}
\end{align}
\textbf{Case UNIT.}Trivial, as $\void$ can only have type 
$\voidtype$. \\
\textbf{Case FAIL.}Trivial, as $\fail{\tau}$ can only have type $\tau$. \\
\textbf{Case DEL.} Trivial, as $\delexp{p}{q}$ can only have type $\aftype{p}{q}$. \\
\textbf{Case LAM.} Given,
\begin{align}
{\TValGpcw{\lamc{x}{τ_1}{\pc'}{e}}{\func{τ_1}{\pc'}{τ_2}}} \label{lu1} \\
{\TValGput{\lamc{x}{τ_1}
{\pc'}{e}}{\func{τ_1}{\pc'}{\mathring{τ_2}}}} \label{lu2} \\
\intertext{inverting \ref{lu1} we get,}
\TVal{Γ,x\ty τ_1;\pc';u}{e}{τ_2} \label{lu3} \\
\rafjudge{Π}{c}{pc} \label{lu4} \\
u= UB(\func{τ_1}{\pc'}{τ_2}) \label{lu5} \\
\rafjudge{Π}{c}{u} \label{lu6} \\
\intertext{inverting \ref{lu2} we get,}
\TVal{Γ,x\ty τ_1;\pc';u'}{e}{\mathring{τ_2}} \label{lu7} \\
\rafjudge{Π}{c}{\mathring{pc}} \label{lu8} \\
u'= UB(\func{τ_1}{\pc'}{\mathring{τ_2}}) \label{lu9} \\ 
\rafjudge{Π}{c}{u'} \label{lu10} \\
\intertext{One thing to notice here, is that $\tau_1$ and $pc'$ is implicitly
defined in the expression $\lamc{x}{τ_1}{\pc'}{e}$, 
so they are same for both \ref{lu1} and \ref{lu2}. 
So it is sufficient to prove that $\tau_2 = \mathring{\tau_2}$. 
From the definition of UB we know that $u$ 
is the upperbound of all the $pc'$s defined in a type, and these $pc'$s 
must be defined in expression $e$. So we can claim that}
u=u' \label{lu11} \\
\intertext{So, \ref{lu7} can be written as}
\TVal{Γ,x\ty τ_1;\pc';u}{e}{\mathring{τ_2}} \label{lu12} \\
\intertext{from IH \ref{lu12} and \ref{lu3} we get}
\tau_2 = \mathring{\tau_2} \label{lu13}
\end{align}
\textbf{Case TLAM.} Similar to case LAM. \\
\textbf{Case APP.} Given,
\begin{align}
{\TValGpcw{e_1~e_2}{τ}} \label{au1} \\
{\TValGput{e_1~e_2}{\mathring{\tau}}} \label{au2} \\
\intertext{inverting \ref{au1} we get}
\TValGpcw{e_1}{\func{τ'}{\pc'}{τ}} \label{au3} \\
\TValGpcw{e_2}{τ'} \label{au4} \\
\stflowjudge*{\pc}{\pc'} \label{au5} \\
\rafjudge{Π}{c}{pc} \label{au6} \\
\intertext{inverting \ref{au2} we get}
\TValGput{e_1}{\func{\mathring{τ'}}{\mathring{\pc}'}{\mathring{τ}}} \label{au7} \\
\TValGput{e_2}{\mathring{τ}'} \label{au8} \\
\stflowjudge*{\mathring{\pc}}{\mathring{\pc}'} \label{au9} \\
\rafjudge{Π}{c}{\mathring{pc}} \label{au6} \\
\intertext{from \ref{au7} we can infer that $e_1$ is a lambda term, and that
implies that $\mathring{\pc}'$ and $\mathring{τ'}$ is a part of the expression
$e_1$. Thus they can not be different from  $pc'$ and $\tau'$ 
in \ref{au3}. Otherwise $e_1$ will
not typecheck. Thus we can write the following}
pc' = \mathring{pc}' \label{au10} \\
\tau' = \mathring{\tau'} \label{au11} \\
\intertext{ so \ref{au7} can be written as }
\TValGput{e_1}{\func{τ'}{\pc'}{\mathring{τ}}} \label{au12} \\
\intertext{from IH \ref{au3} and \ref{au12} we get}
\tau = \mathring{\tau}
\end{align}
\textbf{Case TAPP.} Similar to APP. \\
\textbf{Case PAIR.} Given,
\begin{align}
\TValGpcw{\pair{e_1}{e_2}}{\prodtype{τ_1}{τ_2}} \label{pu1} \\
\TValGput{\pair{e_1}{e_2}}{\prodtype{\mathring{τ_1}}{\mathring{τ_2}}} \label{pu2} \\
\intertext{inverting \ref{pu1} we get} 
\TValGpcw{e_1}{τ_1} \label{pu3} \\
\TValGpcw{e_2}{τ_2} \label{pu4} \\
\rafjudge{\Pi}{c}{pc} \label{pu5} \\
\intertext{inverting \ref{pu2} we get}
\TValGput{e_1}{\mathring{τ_1}} \label{pu6} \\
\TValGput{e_2}{\mathring{τ_2}} \label{pu7} \\  
\rafjudge{\Pi}{c}{\mathring{pc}} \label{pu8} \\
\intertext{from IH,\ref{pu3} and \ref{pu6} we get}
\tau_1 = \mathring{\tau_1} \label{pu9} \\
\intertext{from IH,\ref{pu4} and \ref{pu7} we get}
\tau_2 = \mathring{\tau_2} \label{pu10} \\
\intertext{from \ref{pu9} and \ref{pu10} we, without loss of generality, can say}
\prodtype{τ_1}{τ_2} = \prodtype{\mathring{τ_1}}{\mathring{τ_2}} 
\end{align}
\textbf{Case UNPAIR.} Given,
\begin{align}
{\TValGpcw{\proji{e}}{τ_i}} \label{uu1} \\
{\TValGput{\proji{e}}{\mathring{τ_i}}} \label{uu2} \\
\intertext{inverting \ref{uu1} we get,}
\TValGpcw{e}{\prodtype{τ_1}{τ_2}}  \label{uu3} \\
\rafjudge{Π}{c}{pc} \label{uu4} \\
\intertext{inverting \ref{uu2} we get,}
\TValGput{e}{\prodtype{\mathring{τ_1}}{\mathring{τ_2}}} \label{uu5} \\
\rafjudge{Π}{c}{\mathring{pc}} \label{uu6} \\
\intertext{from IH, \ref{uu3} and \ref{uu5} we get,}
\prodtype{τ_1}{τ_2} = \prodtype{\mathring{τ_1}}{\mathring{τ_2}} \label{uu7} \\
\intertext{without loss of generality we can write,}
\tau_1 = \mathring{\tau_1} \\
\tau_2 = \mathring{\tau_2} \\
\intertext{thus we can say}
\tau_i = \mathring{\tau_i}
\end{align}
\textbf{Case INJ.} Trivial, as the type is annotated in the expression.
\textbf{Case CASE.} Given,
\begin{align}
\TValGpcw{\casexp{e}{x}{e_1}{e_2}{\sumtype{τ_1}{τ_2}}}{τ} \label{cu1} \\
\TValGput{\casexp{e}{x}{e_1}{e_2}{\sumtype{τ_1}{τ_2}}}{\mathring{τ}} \label{cu2} \\
\intertext{inverting \ref{cu1} we get,}
\TValGpcw{e}{\sumtype{τ_1}{τ_2}} \label{cu3} \\
\protjudge*{\pc}{τ} \label{cu4} \\
\rafjudge{Π}{c}{pc} \label{cu5} \\
\TVal{\Pi;Γ,x\ty τ_1;\pc;\worker}{e_1}{τ} \label{cu6} \\
\TVal{\Pi; Γ,x\ty τ_2;\pc;\worker}{e_2}{τ} \label{cu7} \\
\intertext{inverting \ref{cu2} we get, }
\TValGput{e}{\sumtype{\mathring{τ_1}}{\mathring{τ_2}}} \label{cu8} \\
\protjudge*{\mathring{\pc}}{\mathring{τ}} \label{cu9} \\
\rafjudge{Π}{c}{\mathring{pc}} \label{cu10} \\
\TVal{\Pi;Γ,x\ty \mathring{τ_1};\mathring{\pc};\worker}{e_1}{\mathring{τ}} \label{cu11} \\
\TVal{\Pi;Γ,x\ty \mathring{τ_2};\mathring{\pc};\worker}{e_2}{\mathring{τ}} \label{cu12} \\
\intertext{from IH, \ref{cu3} and \ref{cu8}, we get}
\sumtype{τ_1}{τ_2} = \sumtype{\mathring{τ_1}}{\mathring{τ_2}} \label{cu13} \\
\intertext{without loss of generality we can write,}
\tau_1 = \mathring{τ_1} \label{cu14} \\
\tau_2 = \mathring{τ_2} \label{cu15} \\
\intertext{thus using \ref{cu14} and \ref{cu15}, we can 
write \ref{cu11} and \ref{cu12} as }
\TVal{\Pi;Γ,x\ty {τ_1};\mathring{\pc};\worker}{e_1}{\mathring{τ}} \label{cu16} \\
\TVal{\Pi;Γ,x\ty {τ_2};\mathring{\pc};\worker}{e_2}{\mathring{τ}} \label{cu17} \\
\intertext{from IH, \ref{cu6}(or \ref{cu7}) and \ref{cu16} 
(or \ref{cu17}) we get}
\tau = \mathring{\tau}
\end{align}
\textbf{Case UNITM.} Given,
\begin{align}
%\TValGput{\return{ℓ}{e}}{\says{ℓ}{τ}} \label{unu1} \\
%\TValGput{\return{ℓ}{e}}{\says{ℓ}{\mathring{τ}}} \label{unu2} \\
\intertext{inverting \ref{unu1} we get}
\TValGpcw{e}{τ} \label{unu3} \\
\intertext{   ...   }
\intertext{inverting \ref{unu2} we get}
\TValGput{e}{\mathring{τ}} \label{unu4} \\
\intertext{   ....    }
\intertext{from IH, \ref{unu3} and \ref{unu4} we get,}
\tau = \mathring{\tau} 
\end{align}
\textbf{Case UNITL.} Similar to case UNITM.\\
\textbf{Case BINDM.} Given,
\begin{align}
\TValGpcw{\bind{x}{e'}{e}}{τ} \label{bu1} \\
\TValGput{\bind{x}{e'}{e}}{\mathring{τ}} \label{bu2} \\
\intertext{inverting \ref{bu1} we get,}
\TValGpcw{e'}{\says{ℓ  }{τ'}} \label{bu3} \\
\stflowjudge*{\ell \sqcup pc}{\tau} \\
\TVal{Γ,x\ty τ';ℓ  ⊔ j \sqcup pc;\worker}{e}{τ} \label{3.3} \\
\rafjudge{Π}{c}{pc}
\intertext{inverting \ref{bu2} we get,}
\TValGput{e'}{\says{\mathring{ℓ}}{\mathring{τ}'}} \label{bu4} \\
\TVal{Γ,x\ty \mathring{τ}';\mathring{ℓ}⊔ \mathring \sqcup \mathring{pc};c}{e}{\mathring{τ}} \label{bu5} \\
\intertext{...}
\intertext{From \ref{bu4} we know that $e'$ is an eta term, and 
associated $\mathring{\ell}$ are annotated in $e'$ itself, so 
they are same as $\ell$ and $j$ in \ref{bu3}.
Thus \ref{bu4} can be written as }
\TValGput{e'}{\says{ℓ}{\mathring{τ'}}} \label{bu5} \\
\intertext{From IH,\ref{bu3} and \ref{bu5} we get,}
\tau'=\mathring{\tau}'
\intertext{Thus \ref{bu5} can be written as,}
\TVal{Γ,x\ty τ';ℓ⊔ j \sqcup \mathring{pc};c}{e}{\mathring{τ}} \label{6} \\
\intertext{from IH,\ref{3.3}, \ref{6}, we get,}
\tau = \mathring{\tau}
\end{align}
\textbf{Case ASSUME.} Given,
\begin{align}
\TValGpcw{\assume{e}{e'}}{τ} \label{asu1} \\
\TValGput{\assume{e}{e'}}{\mathring{τ}} \label{asu2} \\
\intertext{inverting \ref{asu1} we get,}
\TValGpcw{e}{\aftype{p}{q}} \label{asu3} \\
\TVal{Π,\langle \aftypep{p}{q}\rangle; Γ;\pc;c}{e'}{τ} \label{asu4} \\
\intertext{...}
\intertext{inverting \ref{asu2} we get,}
\TValGput{e}{\aftype{\mathring{p}}{\mathring{q}}} \label{asu5} \\
\TVal{Π,\langle \aftype{\mathring{p}}{\mathring{q}}\rangle; Γ;\mathring{\pc};c}{e'}{\mathring{τ}} \label{asu6} \\
\intertext{...}
\intertext{from IH, \ref{asu3} and \ref{asu5} we can write,}
\aftype{p}{q} = \aftype{\mathring{p}}{\mathring{q}} \label{asu7} \\
\intertext{So we can write \ref{asu6} as }
\TVal{Π,\langle \aftypep{p}{q}\rangle; Γ;\mathring{\pc}}{e'}{\mathring{τ}} \label{asu8}\\
\intertext{from IH, \ref{asu4} and \ref{asu8} we get,}
\tau = \mathring{\tau}
\end{align}
\textbf{Case WHERE.} Similar to case ASSUME. \\
\textbf{Case RUN.} Trivial,as type $\tau$ is annotated in $run$ epression. \\
\textbf{Case RET.} Straightforward using IH. \\
\textbf{Case COMPEND.}\\
\textbf{Case SELECT.}\\
\PM{in macros.tex file change TValGput with pc as 
$\mathring{pc}$ might not be required}

\begin{lemma}[WaitUniqueT]
If~\TValGpcw{E[\expect{\hat{\tau}}]}{\tau}~and~
\TValGput{E[\expect{\hat{\tau}}]}{\tau'}~then~
$\tau=\tau'$.
\end{lemma}
\textbf{Proof.} We prove it by induction over structure of $E$.\\
\textbf{Case E=[ ].} Given,
\begin{align}
\TValGpcw{[~][\expect{\hat{\tau}}]}{\tau} \label{wuh1} \\
\intertext{i.e.}
\TValGpcw{\expect{\hat{\tau}}}{\tau} \label{wuh2} \\
\intertext{and}
\TValGput{[~][\expect{\hat{\tau}}]}{\tau'} \label{wuh3} \\
\intertext{i.e.}
\TValGput{\expect{\hat{\tau}}}{\tau'} \label{wuh4} \\
\intertext{but from WAIT typing rule we know}
\TValP{\Gamma;\hat{\pc};c}{\expect{\hat{\tau}}}{\hat{\tau}} \label{wuh5} \\
\intertext{Thus \ref{wuh2} and \ref{wuh4} typechecks only when}
\tau = \tau' = \hat{\tau}
\end{align}
\textbf{Case E = E e .} Given,
\begin{align}
\TValGpcw{(E e)[\expect{\hat{\tau}}]}{\tau} \label{wuh1} \\
\intertext{i.e.}
\TValGpcw{E[\expect{\hat{\tau}}]e'}{\tau} \label{wuh2} \\
\intertext{and}
\TValGput{(E e)[\expect{\hat{\tau}}]}{\tau'} \label{wuh3} \\
\intertext{i.e.}
\TValGput{E[\expect{\hat{\tau}}]e}{\tau'} \label{wuh4} \\
\intertext{inverting \ref{wuh2} we get}
\TValGpcw{E[\expect{\hat{\tau}}]}{\func{\tau_1}{pc'}{\tau}} \label{wuh5} \\
\TValGpcw{e}{\tau_1} \label{wuh6} \\
\intertext{...}
\intertext{inverting \ref{wuh4} we get}
\TValGput{E[\expect{\hat{\tau}}]}{\func{\mathring{\tau_1}}{\mathring{pc'}}{\tau'}} \label{wuh7} \\
\TValGput{e}{\mathring{\tau_1}} \label{wuh8} \\
\intertext{$\tau_1$ and $\pc'$ are part of syntax so, }
\tau_1 = \mathring{\tau_1} \\
\pc' = \mathring{\pc'} \\
\intertext{so \ref{wuh7} can be written as,}
\TValGput{E[\expect{\hat{\tau}}]}{\func{\tau_1}{pc'}{\tau'}} \label{wuh9} \\
\intertext{From IH, \ref{wuh5}, and \ref{wuh9} we get,}
{\func{\tau_1}{pc'}{\tau}}= {\func{\tau_1}{pc'}{\tau'}}
\intertext{Thus we get,}
\tau = \tau'
\end{align}
\textbf{Case E= $\mathbf{\pair{E}{e}}$}
\PM{Other cases later.}


\begin{lemma}[stackUniqueT]
If~\TValGpcS{s[\hat{\tau}]}{\tau}~and \TValGpcS{s[\hat{\tau}]}{\tau'}~ then~
$\tau=\tau'$. 
\end{lemma}
\textbf{Proof.} The proof is by induction over typing derivation of $s$.\\
\textbf{Case $\mathbf{s=\emptystack[\tau]}$.}  Given,
\begin{align}
\TValGpcS{\emptystack[\hat{\tau}]}{\tau} \label{stuq1} \\
\TValGpcS{\emptystack[\hat{\tau}]}{\tau'} \label{stuq2}\\
\intertext{but from EMPTY typing rule we know,}
\TValGpcS{\emptystack[\hat{\tau}]}{\hat{\tau}} \\
\intertext{so \ref{stuq1} and \ref{stuq2} typechecks only when}
\tau=\tau'=\hat{\tau}
\end{align}
\textbf{Case $\mathbf{s = \distS{E[\expect{\hat{\tau}}]}{c}{s[\hat{\tau}]}}$.} 
Given,
\begin{align}
\TValGpcS{\distS{E[\expect{\hat{\tau}}]}{c}{s[\hat{\tau}]}}{\tau} \label{stuq3} \\
\TValGpcS{\distS{E[\expect{\hat{\tau}}]}{c}{s[\hat{\tau}]}}{\tau'} \label{stuq4} \\
\intertext{inverting \ref{stuq3} we get,}
\TValGpcw{E[\expect{\hat{\tau}}]}{\tau''} \label{stuq5} \\
\TValGpcS{s[\tau'']}{\tau} \label{stuq6} \\
\stflowjudge*{\pc}{\pc'} \\
\rafjudge{Π}{c}{pc} \\
\intertext{inverting \ref{stuq4} we get}
\TValGpcw{E[\expect{\hat{\tau}}]}{{\tau_1}''} \label{stuq7} \\
\TValGpcS{s[{\tau_1}'']}{\tau'} \label{stuq8} \\
\intertext{from \ref{stuq5} and \ref{stuq7} and lemma WaitUniqueT,}
\tau'' = {\tau_1}'' \\
\intertext{Thus \ref{stuq8} can be written as}
\TValGpcS{s[\tau'']}{\tau'} \label{stuq9} \\
\intertext{from IH, \ref{stuq6} and \ref{stuq9} we get}
\tau = \tau'
\end{align}
\PM{NOTE:- in WaitUnique and stackUniqueT lemma pc is same for both.}

\begin{lemma}[distUniqueT]
If~\TValGpcS{\distcon{e}{c}{s}}{\tau}~and~ \TValGpcS{\distcon{e}{c}{s}}{\tau'}
~then~ $\tau =\tau'$.
\end{lemma}
\textbf{Proof.}
Straightforward proof using lemmas UniqueType and
WaitUniqueT.

\begin{lemma}[$\Gamma$-Weakening]
If \TValGpcw{e}{\tau} and for 
all $\tau'$ and $x \notin dom(\Gamma)$,
\TValP{\Gamma,x\ty\tau';pc;c}{e}{\tau}
\end{lemma}
\textbf{Proof.} 
By Induction on structure of $e$.


\begin{lemma}[CTX]
\PM{subject reduction uses CTX lemma, and CTX lemma needs uniquetypes ?}
If \TValGpcw{E[e]}{\tau} and 
$x\notin dom(\Gamma)$ then
$\exists \tau'$, such that 
\TValP{\Gamma,x\ty\tau',\pc;c}{E[x]}{\tau} and \TValGpcw{e}{\tau'}.
\end{lemma}
\textbf{Proof.}
By induction on structure of $E$.\\
Given,
\begin{equation}\label{300}
\TValGpcw{E[e]}{\tau}
\end{equation}
and we need to prove 
\begin{align}
\exists{\tau'}.\TValGpcw{e}{\tau'} \label{299} \\
\intertext{and}
\TValP{\Gamma,x\ty\tau',\pc;c}{E[x]}{\tau} \label{298}
\end{align}
\textbf{Case E=[ ].}  
Applying definition Plug in \ref{300} 
(i.e.$[~][e] = e$) we get 
\begin{align}
\TValGpcw{e}{\tau} \label{300.5} \\
\intertext{and $\exists{\tau}$ in \ref{299} we get}
\TValGpcw{e}{\tau} \label{301}\\ 
\intertext{from VAR rule for variable $x$}
\TValP{\Gamma,x\ty\tau,\pc;c}{x}{\tau} \label{302} \\
\intertext{from \ref{302} and definition Plug}
\TValP{\Gamma,x\ty\tau,\pc;c}{[~][x]}{\tau} \label{303} \\
\end{align}
{\ref{301} and \ref{303} together proves \ref{299} and \ref{298}}. \\
\textbf{Case E = E e'} 
Applying definition Plug in \ref{300} ($(E e')[e] = E[e] e'$)
\begin{align}
\TValGpcw{E[e]e'}{\tau} \label{304} \\
\intertext{inverting \ref{304}}
\TValGpcw{E[e]}{\func{\tau_1}{pc'}{\tau}} \label{305} \\
\intertext{and}
\TValGpcw{e'}{\tau_1} \label{306} \\
\intertext{inverting \ref{306} using $\Gamma$-Weakening }
\TValP{\Gamma,x\ty\tau',\pc;c}{e'}{\tau_1} \label{307} \\
\intertext{I.H. on \ref{305} we get (destruct IH with $\tau'$)}
\TValGpcw{e}{\tau'} \label{308} \\
\intertext{and}
\TValP{\Gamma,x\ty\tau',\pc;c}{E[x]}{\func{\tau_1}{pc'}{\tau}} \label{309} \\
\intertext{from \ref{309}, \ref{307} and APP we get}
\TValP{\Gamma,x\ty\tau',\pc;c}{E[x]e'}{\tau} \label{310}
\end{align}
{\ref{308} and \ref{310} together proves \ref{299} and \ref{298}}\\ 
\textbf{Case E=$\pair{E}{e'}$}.
Applying definition Plug in \ref{300} ($(\pair{E}{e'})[e] = \pair{E[e]}{e'}$)
and $\tau = \prodtype{\tau_3}{\tau_4}$
\begin{align}
\TValGpcw{\pair{E[e]}{e'}}{\tau} \label{311} \\
\intertext{inverting \ref{311} }
\TValGpcw{E[e]}{\tau_3} \label{312} \\
\intertext{and}
\TValGpcw{e'}{\tau_4} \label{313} \\
\intertext{inverting \ref{312} \ref{313} using $\Gamma$-Weakening }
\TValP{\Gamma,x\ty\tau',\pc;c}{E[e]}{\tau_3} \label{314} \\
\intertext{and}
\TValP{\Gamma,x\ty\tau',\pc;c}{e'}{\tau_4} \label{315}
\intertext{I.H. on \ref{312} we get(destruct IH with $\tau'$)}
\TValGpcw{e}{\tau'} \label{316} \\
\intertext{and}
\TValP{\Gamma,x\ty\tau',\pc;c}{E[x]}{\tau_3} \label{317} \\
\intertext{from \ref{317}, \ref{315} and PAIR rule we get}
\TValP{\Gamma,x\ty\tau',\pc;c}{\pair{E[x]}{e'}}{\tau} \label{318}
\end{align}
{\ref{316} and \ref{318} together proves \ref{299} and \ref{298}}\\
\textbf{Case E=$\proji{E}$}
applying definition Plug in \ref{300} ($(\proji{E})[e] = \proji{E[e]}$)
and $\exists.\tau'$
\begin{align}                                                     
\TValGpcw{\proji{E[e]}}{\tau_i} \label{320} \\
\intertext{inverting \ref{320} and $\tau = \tau_i, i \in \{3,4\}$}
\TValGpcw{E[e]}{\prodtype{\tau_3}{\tau_4}} \label{321} \\
\intertext{applying $\Gamma$-Weakening in \ref{321}}
\TValP{\Gamma,x\ty\tau',\pc;c}{E[e]}{\prodtype{\tau_3}{\tau_4}} \label{322} \\
\intertext{I.H. in \ref{322} we get}
\TValGpcw{e}{\tau'} \label{323} \\
\intertext{and}
\TValP{\Gamma,x\ty\tau',\pc;c}{E[x]}{\prodtype{\tau_3}{\tau_4}} \label{324} \\
\intertext{from \ref{324}, and UNPAIR rule we get}
\TValP{\Gamma,x\ty\tau',\pc;c}{\proji{E[x]}}{\tau} \label{325}
\end{align}
{\ref{323} and \ref{325} together proves \ref{299} and \ref{298}}\\
\textbf{Case E= $\assume{E}{e'}$} 
\PM{(in all of the cases above
 inversion will also have a c actsfor pc premise that has to be added later.)}


\begin{lemma}[WAIT]
If \TValGpcw{E[e]}{\tau} and \TValGpcw{e}{\tau'} then 
\TValGpcw{E[\expect{\tau'}]}{\tau}.
\end{lemma}
\textbf{proof.} Proof by induction over structure of E. \\
Given,
\begin{align}
\TValGpcw{E[e]}{\tau} \label{expect1} \\
\TValGpcw{e}{\tau'}  \label{expect2} \\
\intertext{need to prove} 
\TValGpcw{E[\expect{\tau'}]}{\tau} \label{expect3}\\
\end{align}
\textbf{Case E =[ ]}.
From \ref{expect1} we get
\begin{align}
\TValGpcw{[~][e]}{\tau} \\
\intertext{i.e.}
\TValGpcw{e}{\tau} \label{expect4} \\
%\intertext{from \ref{expect2}, \ref{expect4} and UniqueType lemma we get}
\intertext{but from \ref{expect2}}
\TValGpcw{e}{\tau'} \label{expect4.5}
\intertext{thus}
\tau = \tau'  \label{expect5} \\
\intertext{from WAIT typing rule we get}
\TValGpcw{\expect{\tau'}}{\tau'}  \label{expect6}  \\
\intertext{but $\tau=\tau'$, so \ref{expect6} can be written as}
\TValGpcw{\expect{\tau'}}{\tau} \label{expect7} \\
\intertext{\ref{expect7} can be written as }
\TValGpcw{[~]\expect{\tau'}}{\tau} 
\end{align}
\textbf{Case E = E e'}.
from \ref{expect1} we get
\begin{align}
\TValGpcw{(E~e')[e]}{\tau} \\
\intertext{i.e.}
\TValGpcw{E[e]e'}{\tau} \label{expect8} \\
\intertext{inverting \ref{expect8} we get}
\TValGpcw{E~[e]}{\func{\tau_1}{pc'}{\tau}} \label{expect9} \\
\TValGpcw{e'}{\tau_1} \label{expect10} \\
\rafjudge{\Pi}{c}{pc} \label{expect10.5} \\
\rflowjudge{\Pi}{pc}{pc'} \label{expect10.6} \\
\intertext{from \ref{expect2}, \ref{expect9} and IH we get}
\TValGpcw{E~[\expect{\tau'}]}{\func{\tau_1}{pc'}{\tau}} \label{expect11} \\
\intertext{from \ref{expect10},\ref{expect11},\ref{wait10.5},\ref{wait10.6} and APP typing rule we get}
\TValGpcw{E[\expect{\tau'}]e'}{\tau} \label{expect12} \\
\intertext{\ref{expect12} can be written as}
\TValGpcw{(E~e')[\expect{\tau'}]}{\tau} 
\end{align}
\PM{Other cases for E later.}


\begin{lemma}[RWAIT]
\TValGpcw{v}{\tau'} and
\TValGpcw{E[\expect{\tau'}]}{\tau} then
\TValGpcw{E[v]}{\tau}.
\end{lemma}
\PM{both expect and rexpect can be generalized with e of type tau' instead of wait term
and to be used in subject reduction for case E-Step.}
\PM{Bracketed subject reduction failed bcz of pc' in select.}
\textbf{proof.} Proof by induction over structure of E. \\
Given,
\begin{align}
\TValGpcw{v}{\tau'}  \label{rexpect1} \\
\TValGpcw{E[\expect{\tau'}]}{\tau} \label{rexpect2} \\
\intertext{need to prove}
\TValGpcw{E[v]}{\tau} \label{rexpect3}\\
\end{align}
\textbf{Case E =[ ]}.
from \ref{rexpect2} we get
\begin{align}
\TValGpcw{[~][\expect{\tau'}]}{\tau} \\
\intertext{i.e.}
\TValGpcw{\expect{\tau'}}{\tau} \label{rexpect4} \\
\intertext{from WAIT rule we get }
\TValGpcw{\expect{\tau'}}{\tau'}  \label{rexpect5} \\
\intertext{but from \ref{rexpect4}, $\TValGpcw{\expect{\tau'}}{\tau}$, thus}
\tau=\tau' \label{rexpect6} \\
\intertext{Thus we can write \ref{rexpect1} as }
\TValGpcw{v}{\tau}  \label{rexpect7} \\
\intertext{from definition Plug, \ref{rexpect7} can be written as}
\TValGpcw{[~]v}{\tau} 
\end{align}
\textbf{Case E =E e'}.
from \ref{rexpect2} we get
\begin{align}
\TValGpcw{(E~e')[\expect{\tau'}]}{\tau} \\
\intertext{i.e.}
\TValGpcw{E[\expect{\tau'}]e'}{\tau} \label{rexpect8} \\
\intertext{inverting \ref{rexpect8} we get}
\TValGpcw{E~[\expect{\tau'}]}{\func{\tau_1}{pc'}{\tau}} \label{rexpect9} \\
\TValGpcw{e'}{\tau_1} \label{rexpect10} \\
\rafjudge{\Pi}{c}{pc} \label{rexpect10.5} \\
\rflowjudge{\Pi}{pc}{pc'} \label{rexpect10.6} \\
\intertext{from \ref{rexpect1}, \ref{rexpect9} and IH we get}
\TValGpcw{E~[v]}{\func{\tau_1}{pc'}{\tau}} \label{rexpect11} \\
\intertext{from \ref{rexpect10}, \ref{rexpect11}, \ref{rwait10.5}, \ref{rwait10.6} and APP typing rule we get}
\TValGpcw{E[v]e'}{\tau} \label{rexpect12} \\
\intertext{\ref{rexpect12} can be written as}
\TValGpcw{(E~e')[v]}{\tau}
\end{align}
\PM{Other cases for E later.}
\PM{runnable gives me info about pcs but not  
if e will be typechecked as well, so used UB}



\begin{lemma}[Values are typable at any host and any \pc]
Let $\TValGpcw{v}{\tau}$. If  
$\rafjudge{\Pi}{c'}{pc'}$ and $\rafjudge{\Pi}{c'}{UB(\tau)}$ 
then $\TValGpcdashpc{v}{\tau}$.
\end{lemma}
\textbf{Proof.} Given that,
\begin{align}
\rafjudge{\Pi}{c'}{pc'}   \label{VTH} \\
\rafjudge{\Pi}{c'}{UB(\tau)}  \label{VTH.5} 
\end{align}
Using induction over values.\\ \\
%\textbf{case FAIL.} Using [FAIL] and (\ref{VTH}) we have 
%{\TValGpcdashpc{\fail{\says{\ell}{\tau}}}{\says{\ell}{\tau}}}\\ \\
\textbf{Case UNITM.} Using [UNIT] and (\ref{VTH}) we have  
{\TValGpcdashpc{\void}{\voidtype}}\\ \\
\textbf{Case DEL.} Using [DEL] and (\ref{VTH}) we have 
{\TValGpcdashpc{\delexp{p}{q}}{\aftype{p}{q}}}\\ \\
\textbf{Case PAIR.} Given 
\begin{equation}\label{pairVTH}
{\TValGpcw{\pair{v_1}{v_2}}{\prodtype{τ_1}{τ_2}}}
\end{equation}
Inverting (\ref{pairVTH}) we get 
\begin{equation}\label{pairVTH1}
\TValGpcw{v_1}{τ_1}
\end{equation} 
and
\begin{equation}\label{pairVTH2}
\TValGpcw{v_2}{τ_2}
\end{equation}
By applying induction hypothesis on (\ref{pairVTH1}) and (\ref{pairVTH2}), we get
\begin{equation}\label{pairVTH3}
\TValGpcdashpc{v_1}{τ_1}
\end{equation}
and
\begin{equation}\label{pairVTH4}
\TValGpcdashpc{v_2}{τ_2}
\end{equation}
From rule [PAIR], (\ref{pairVTH3}), (\ref{pairVTH4}), and (\ref{VTH}) we get
{\TValGpcdashpc{\pair{v_1}{v_2}}{\prodtype{τ_1}{τ_2}}}\\ \\
\textbf{Case INJ.} Similar to \textbf{case PAIR}.\\ \\
\textbf{Case UNITL.} Given 
\begin{equation}\label{sealVTH}
{\TValGpcw{\returnv{ℓ }{v}}{\says{ℓ }{τ}}}
\end{equation}
Inverting (\ref{sealVTH}) we get
\begin{equation}\label{sealVTH1}
 \TValGpcw{v}{\tau} 
\end{equation}
By applying induction hypothesis on (\ref{sealVTH1}) we get
\begin{equation}\label{sealVTH2} 
\TValGpcdashpc{v}{τ}
\end{equation}
Thus from rule [SEAL], (\ref{VTH}) and (\ref{sealVTH2}) we get 
{\TValGpcdashpc{\returnv{ℓ  }{v}}{\says{ℓ  }{τ}}}\\ \\
\textbf{Case WHERE.} Given
\begin{equation}\label{whereVTH}
{\TValGpcw{(\where{v_1}{v_2})}{τ}}{}
\end{equation}
Inverting (\ref{whereVTH}) we get
\begin{equation}\label{whereVTH1}
 \TValGpcw{v_2}{\aftype{p}{q}}
\end{equation}
\begin{equation}\label{whereVTH2}
\rafjudge{Π}{\pcmost}{\voice{q}}
\end{equation}
\begin{equation}\label{whereVTH3}
\rafjudge{Π}{\voice{\confid{p}}}{\voice{\confid{q}}}
\end{equation}
\begin{equation}\label{whereVTH4}
\TVal{Π,\langle \aftypep{p}{q}\rangle;Γ;\pc}{v_1}{τ}
\end{equation}
By applying induction hypothesis on (\ref{whereVTH1}) and
and (\ref{whereVTH4}) and $\Pi$-extension rule on 
(\ref{whereVTH4}) we get
\begin{equation}\label{whereVTH5}
 \TValGpcdashpc{v_2}{\aftype{p}{q}}
\end{equation}
\begin{equation}\label{whereVTH6}
\TVal{Π,\langle \aftypep{p}{q}\rangle;Γ;\pc';c'}{v_1}{τ}
\end{equation}
Thus from rule [WHERE], (\ref{VTH}), (\ref{whereVTH2}),  (\ref{whereVTH3}),
 (\ref{whereVTH5}) and (\ref{whereVTH6}) we get
{\TValGpcdashpc{(\where{v_1}{v_2})}{τ}}{} \\ \\
\textbf{Case LAM.} We have 
\begin{equation}\label{lamVTH}
{\TValGpcw{\lamc{x}{τ_1}{\pc''}{e}}{\func{τ_1}{\pc''}{τ_2}}}
\end{equation}
Inverting (\ref{lamVTH}) we get
\begin{align}
\TVal{\Pi;Γ,x\ty τ_1;\pc'';u}{e}{τ_2}  \label{lamVTH1} \\
u = UB(\func{τ_1}{\pc''}{τ_2}) \label{lamVTH1.1} \\
\rafjudge{\Pi}{c}{pc}  \\ 
\rafjudge{\Pi}{c}{u}  \\
\intertext{given in lemma statement} 
\rafjudge{\Pi}{c'}{pc'}  \label{lamvth1.2} \\                                  
\rafjudge{\Pi}{c'}{u}  \label{lamvth1.3} \\ 
\intertext{from \ref{lamvth1.2}, \ref{lamvth1.3}, 
\ref{lamVTH1} and LAM rule we get}
{\TValGpcdashpc{\lamc{x}{τ_1}{\pc''}{e}}{\func{τ_1}{\pc''}{τ_2}}} \\ \\
\end{align}
\textbf{Case TLAM.} Similar to \textbf{case LAM}.\\ \\
%\textbf{Case FAIL.} We have 
%\begin{equation}\label{failVTH}
%\TValGpcw{\fail{\tau}}{\tau}
%\end{equation} 
%From (\ref{VTH}) and [FAIL] rule we have \TValGpcdashpc{\fail{\tau}}{\tau}  
\textbf{Case BRACKET.}
Given,
\begin{align}
{\TValGpcw{\bracket{w_1}{w_2}}{\tau}}  \label{br1} \\
\intertext{inverting \ref{br1} we get}
\protjudge{\delegcontext}{H^\pi}{\tau} \label{br2} \\
\TValGpcw{w_1}{\tau} \label{br3} \\      
\TValGpcw{w_2}{\tau} \label{br4} \\
\intertext{IH on \ref{br3} and \ref{br4}}
\TValGpcdashpc{w_1}{\tau} \label{br5} \\
\TValGpcdashpc{w_2}{\tau} \label{br6} \\
\intertext{from \ref{br5},\ref{br6} and \ref{br2} we get}
{\TValGpcdashpc{\bracket{w_1}{w_2}}{\tau}}
\end{align}

\begin{lemma}[$\pc$ reduction]
 Let \TValGpcw{e}{\tau}.
  For all $\pc, \pc'$, such that
  \rflowjudge{\delegcontext}{\pc'}{\pc} and 
  \rafjudge{Π}{c}{pc'} (and 
\rflowjudge{\delegcontext}{\j(\tau)}{pc'}) then
  \TValP{\varcontext;pc';c}{e}{\tau} holds.
\end{lemma}
\textbf{Proof.}
Using induction. 
\PM{Works for run /ret after adding pc' in run. But
breaks for compEnd and select after incorporating j, as 
j might not flow to reduced pc. So to fix it 
$(\rflowjudge{\delegcontext}{\j(\tau)}{pc'})$ is added
in the lemma statement.
works for bracket/bracket-values}

%\begin{lemma}[CTXpc]
%If~ \TValP{\varcontext;pc';c}{e}{\tau} , $\forall{v}.${\TValGpcw{v}{\tau'}} and
%{\TValGpcw{E[v]}{\tau}}, and \rflowjudge{\delegcontext}{\pc}{\pc'} 
%then {\TValGpcw{E[e]}{\tau}}
%\end{lemma}
%\textbf{Proof.}
%First apply pc reduction in \TValP{\varcontext;pc';c}{e}{\tau}
%with $pc$, then
%apply Lemma CTX.
%

\begin{lemma}[$\Pi$ extension]
If {\TValGpc{e}{\tau}} then 
\TVal{\delegcontext\delegconcat \delexp{p}{q};\varcontext;\pc}{e}{\tau} 
for any $p, q \in \L$.
\end{lemma}
\begin{lemma}[Robust assumption]
 If
        $\rafjudge{\Pi}{\pc}{\voice{b}}$, then
        $\rafjudge{\delegcontext\delegconcat \delexp{a}{b}}{\pc}{\voice{b}}$ for any $a, b \in \L$.
\end{lemma}
\begin{lemma}[Robust Protection]
If
        $\protjudge{\delegcontext}{\pc}{\tau}$,
        then
        $\protjudge{\delegcontext\delegconcat \delexp{a}{b}}{\pc}{\tau}$ for any $a, b \in \L$.
\end{lemma}
\begin{lemma}[Variable substitution]
If \TValP{\Gamma,x:\tau';\pc;c}{e}{\tau} and \TValGpcw{w}{τ'}, then \TValGpcw{\subst{e}{x}{w}}{\tau}.
\end{lemma}
%\begin{proof}
trivial
%\end{proof}
\textbf{Proof.}
Proof is by induction on the typing derivation of $e$.
\\
\textbf{Case LAM.}
Given,
\begin{align}
\TValGpcw{w}{τ'} \label{99} \\
{\TValP{\Gamma,x'\ty \tau';\pc;c}{\lamc{x}{τ_1}{\pc'}{e}}{\func{τ_1}{\pc'}{τ_2}}} \label{100} \\
{\TValP{Γ,x'\ty \tau',x\ty τ_1;\pc';c'}{e}{τ_2}} \label{101} \intertext{inverting \ref{100}}\\
\rafjudge{\Pi}{c'}{\pc'} \label{102} \intertext{clearance of \ref{101}} \\
{\TValGpcdashpc{w}{τ'}} \label{103} \intertext{lemma values host pc and \ref{102}} \\
{\TValP{Γ,x\ty τ_1;\pc';c'}{\subst{e}{x'}{w}}{τ_2}} \label{104} \intertext{IH, \ref{101}, \ref{103}}
\TValGpcw{\lamc{x}{τ_1}{\pc'}{\subst{e}{x'}{w}}}{\tau_2} \intertext{from LAM, and \ref{104}}
\end{align}
\textbf{Case RUN.} Same as before.
\begin{lemma}[Type substitution]
 %Let $\tau'$ be well-formed in $\varcontext, X, \varcontext'$.                     If $\TValP{\varcontext, X, \varcontext';\pc}{e}{\tau}$ then $\TValP{\varcontext, \varcontext'[X \mapsto \tau'];\pc}{e[X \mapsto \tau']}{\tau [X \mapsto \tau']}$.
\end{lemma}
\begin{lemma}[Soundness]\label{lemma:sound}
        If  $e \stepsone e'$ then
        $\outproj{e}{k} \stepsto \outproj{e'}{k}$ for $k ∈ \{1,2\}$.
 \end{lemma}
\textbf{Proof}
By induction on the evaluation of $e$.  All bracketed
rules in Figure~\ref{fig:brackets} except 
\ruleref{B-Step} only expand brackets, so
$\outproj{e}{k} = \outproj{e'}{k}$ for $k \in \{1,2\}$. 
For \ruleref{B-Step}, $\outproj{e}{i} \stepsone \outproj{e'}{i}$  
and $\outproj{e}{j} = \outproj{e'}{j}$ .

\begin{lemma}[Subject Reduction(within a host)]
Let  \TValGpcw{e}{τ} \rafjudge{\Pi}{c}{UB(\tau)}.  If $e \stepsone e'$
%\rafjudge{\delegcontext}{c}{\pc}  
then \TValGpcw{e'}{τ}.
\end{lemma}
\textbf{Proof.} \newline
\textbf{Case E-BINDM.} Given $e={\bind{x}{\returnv{ℓ}{w}}{e_1}}$ and 
$e'= \subst{e_1}{x}{w}$ and also
\begin{equation}\label{bind1}
{\TValGpcw{\bind{x}{\returnv{ℓ}{w}}{e_1}}{τ}}
\end{equation}
From the premises of (\ref{bind1}) we have
\begin{align}
  \TValGpcw{\returnv{ℓ}{w}}{\says{\ell}{\tau'}} \label{eq:sbindm0} \\
  \TValGpcw{w}{\tau'} \label{eq:sbindm1} \\
  \TVal{\Pi;\Gamma,x:\tau';\pc \sqcup \ell \sqcup j;c}
    {e_1}{\tau} \label{eq:sbindm2} \\
  \protjudge{\delegcontext}{\pc \sqcup \ell}{\tau} \label{eq:sbindm4} \\
  \rafjudge{\delegcontext}{c}{\pc}
\end{align}
(Since $\rafjudge{\delegcontext}{c}{\pc}$) applying $\pc$ reduction lemma in 
(\ref{eq:sbindm2})(23) we get $\TValP{\Gamma, x:\tau'; \pc}{e_1}{\tau} \label{eq:sbindm2}$.
Invoking variable substitution lemma, we thus have 
${\TValGpcw{\subst{e_1}{x}{w}}{\tau}}$.
\PM{j flowsto J(tau) is not used and as of now
nothing breaks without it.}
\\ \\
\textbf{Case E-RETSTEP}
\begin{equation}\label{retcon}
{\ret{e_1}{c'}} \stepsone {\ret{e_1'}{c'}}
\end{equation}
Given, $e=\ret{e_1}{c'}$ and $e'=\ret{e_1'}{c'}$ and also
\begin{equation}\label{ret1}
{\TValGpcw{\ret{e_1}{c'}}{\tau}}
\end{equation} 
From the premises of (\ref{ret1}) we get
\begin{align}
{\TValGpcw{e_1}{\tau}}  \label{ret1pr1} \\
\rafjudge{\delegcontext}{c}{\pc}  \label{ret1pr2}
\end{align}
and applying induction hypothesis on the premise of (\ref{retcon})
we get 
\begin{equation}\label{ret1IH}
{\TValGpcw{e_1'}{\tau}}
\end{equation}
From (\ref{ret1IH}) and (\ref{ret1pr2}) we have
${\TValGpcw{\ret{e_1'}{c'}}{\tau}}$ \\ \\
\textbf{E-COMPEND} Given,
$e=(\compEnd{\tau}{\returnv{\ell_1}{w}}{\returnv{\ell_2}{w}})$ and  
$e'=\returnv{\ell_1 \wedge \ell_2}{w}{j_1 \sqcup j_2}$
and also, 
\begin{equation}\label{ce1}
\TValGpcw{\compEnd{\tau}{\returnv{\ell_1}{w}}{\returnv{\ell_2}{w}}}
{\says{\ell_1 \wedge \ell_2}{\tau}{j_1 \sqcup j_2}} 
\end{equation}
Inverting (\ref{ce1})
\begin{align}
\TValGpcw{\returnv{\ell_1}{w}}{\says{\ell_1}{τ}} \label{ce2}\\
\TValGpcw{\returnv{\ell_2}{w}}{\says{\ell_2}{τ}} \label{ce3}\\
\confid{c} \rhd \says{\ell_1}{\tau} \label{ce4}\\
\confid{c} \rhd \says{\ell_2}{\tau}  \label{ce5}\\
\rafjudge{\Pi}{c}{pc} \label{ce6}\\
%\stflowjudge*{pc} \label{ce7}\\
%\stflowjudge*{pc}  \label{ce7}\\
\end{align}
Inverting (\ref{ce2}) or (\ref{ce3}) further, we get
\begin{equation}\label{ce8}
\TValGpcw{w}{\tau}
\end{equation}
From rule [UNITL], (\ref{ce8}) and (\ref{ce6}) we can say
$\TValGpcw{\returnv{\ell_1 \wedge \ell_2}{w}{j_1 \sqcup j_2}}
{\says{\ell_1 \wedge \ell_2}{\tau}{j_1 \sqcup j_2}}$ \\ \\
\textbf{Case E-COMPAREFAIL.} Trivial, as $\fail{\tau}$ typechecks
with any protected type, and based on our type-system $\tau$ is always 
a protected type.\\ \\
\textbf{Case E-SELECT.}
Given,
$e = \select{\returnv{\ell_1}{w_1}}{\returnv{\ell_2}{w_2}}{\ell}$ 
and 
$e'= \returnv{\ell}{w_1}$ 
where $\ell = \ell_1 \sqcup \ell_2$ and $j=j_1 \sqcup j_2$.
The following is also given. 
\begin{equation}\label{sel1}
{\TValGpcw{\select{\returnv{\ell_1}{w_1}}{\returnv{\ell_2}{w_2}}{\ell}}{\says{\ell}{τ}}}
\end{equation}
Inverting (\ref{sel1}) we get the following,
\begin{align}
\TValGpcw{\returnv{\ell_1}{w_1}}{\says{\ell_1}{τ}} \label{se2}\\
\TValGpcw{\returnv{\ell_2}{w_2}}{\says{\ell_2}{τ}} \label{se3}\\
\rafjudge{Π}{c}{pc} \label{se4}\\
%\stflowjudge*{\pc} \label{se5} \\
%\stflowjudge*{\pc} \label{se6}
\end{align}
Further inverting (\ref{se2}) we get,
\begin{equation}\label{se7}
\TValGpcw{w_1}{\tau}  
\end{equation} 
Thus from rule [UNITL], (\ref{se7}) and (\ref{se4}) we can argue,
$\TValGpcw{\returnv{\ell}{w_1}}{\says{\ell_1}{τ}}$ \\ \\
\textbf{Case E-SELECTOR.} Similar to above. The only diffrence is we need to
invert both (\ref{se2}) and (\ref{se3}) 
and argue both $\TValGpcw{\returnv{\ell}{w_1}}{\tau}$
and $\TValGpcw{\returnv{\ell}{w_2}}{\tau}$ holds.\\ \\                    
\textbf{Case E-SELECTFAIL.} Trivial, as $\fail{\tau}$ typechecks for any
protected type, and based on our type-system $\tau$ is always
a protected type.\\ \\. \\ \\
\textbf{CasecW-COMPEND1} \PM{not sure if this required or the Pi extension works}
Given 
${\config*{\compEnd{\tau}{(\where{w}{v})}{e}}} \stepsone
{\config*{\where{(\compEnd{\tau}{w}{e})}{v}}}$ where
$e=\compEnd{\tau}{(\where{w}{v})}{e}$ and 
$e'=\where{(\compEnd{\tau}{w}{e})}{v}$ and
\begin{equation}\label{cew}
{\TValGpcw{\compEnd{\tau}{\where{w}{v}}{e}}{\says{(\ell_1 \wedge \ell_2 )}
{\tau}{j_1 \sqcup j_2}}}
\end{equation}
From the premises of (\ref{cew}) we get the following
\begin{align}
\TValGpcw{\where{w}{v}}{\says{\ell_1}{τ}} \label{cew1}\\
\TValGpcw{e}{\says{\ell_2}{τ}} \label{cew2}\\
\confid{c} \rhd \says{\ell_1}{\tau} \label{cew3}\\
\confid{c} \rhd \says{\ell_2}{\tau} \label{cew4}\\
\rafjudge{\Pi}{c}{pc}  \label{cew5} \\
\end{align}
Further inverting (\ref{cew1}) we get
\begin{align}
\TValGpcw{v}{\aftype{p}{q}}\label{cew8} \\
\rafjudge{Π}{\pcmost}{\voice{q}} \label{cew9} \\
\rafjudge{Π}{\voice{\confid{p}}}{\voice{\confid{q}}} \label{cew10}\\
\TVal{Π,\langle \aftypep{p}{q}\rangle;Γ;\pc}{w}
{\says{\ell_1}{τ}} \label{cew11}
\end{align}
We need to prove that 
\begin{equation}\label{cew12}
\TValGpcw{\where{(\compEnd{\tau}{w}{e})}{v}}{\says{(\ell_1 \wedge \ell_2 )}
{\tau}{j_1 \sqcup j_2}}
\end{equation}
From  (\ref{cew3}),(\ref{cew4}),(\ref{cew5}), (\ref{cew6}), (\ref{cew7}),
(\ref{cew11}), and extending delegation context in (\ref{cew2}) we can prove 
\begin{equation}\label{cew13}
\TVal{Π,\langle \aftypep{p}{q}\rangle;Γ;\pc}
{\compEnd{\tau}{w}{e}}{\says{(\ell_1 \wedge \ell_2 )}
{\tau}{j_1 \sqcup j_2}}
\end{equation}
From (\ref{cew8}),(\ref{cew9}),(\ref{cew10}),(\ref{cew13}),  
and [WHERE] rule we can argue (\ref{cew12}) is true. \\ \\
\textbf{Case W-COMPEND2.} Similar to the above case.\\
\textbf{Case W-SELECT1.} Similar to  \textbf{case W-COMPEND1}.\\
\textbf{Case W-SELECT2.} Similar to \textbf{case W-COMPEND1}.\\ \\
\textbf{Case B-SELECT.} Given,
\begin{align}
{\select{\bracket{w_1}{w_2}}{\bracket{w_1'}{w_2'}}{\tau}} 
& \stepsone 
& {\bracket{\select{w_1}{w_1'}{\tau}}{\select{w_2}{w_2'}{\tau}}} \label{bse1} 
\end{align}
\begin{align}
\TValGpcw{{\select{\bracket{w_1}{w_2}}{\bracket{w_1'}{w_2'}}{\tau}}}{\tau} \label{bse2}\\
\intertext{ We need to prove}
\TValGpcw{\bracket{\select{w_1}{w_1'}{\tau}}{\select{w_2}{w_2'}{\tau}}}{\tau}
\intertext{inverting \ref{bse2} we get(here $\ell = {\ell_1 \join \ell_2}
j={j_1 \join j_2}$)}
\TValGpcw{\bracket{w_1}{w_2}}{\says{\ell_1}{τ}} \label{bse3} \\
\TValGpcw{\bracket{w_1'}{w_2'}}{\says{\ell_2}{τ}} \label{bse4} \\
\rafjudge{Π}{c}{pc} \label{bse5} \\
\intertext{inverting \ref{bse3} and \ref{bse4}}
\TValGpcw{w_1}{\says{\ell_1}{τ}} \label{bse8} \\
\TValGpcw{w_2}{\says{\ell_1}{τ}} \label{bse9} \\
\TValGpcw{w_1'}{\says{\ell_2}{τ}} \label{bse10} \\
\TValGpcw{w_2'}{\says{\ell_2}{τ}} \label{bse11} \\
\stflowjudge*{\H^\pi}{\ell_1} \label{bse11.1} \\
\stflowjudge*{\H^\pi}{\ell_2} \label{bse11.2} \\
\intertext{from \ref{bse11.1} and \ref{bse11.2} we get}
\stflowjudge*{\H^\pi}{\ell_1 \join \ell_2} \label{bse11.3} \\
\intertext{applying values pc in \ref{bse8}, \ref{bse9}, \ref{bse10}
\ref{bse11} with pc' as
$pc \join \ell_1 \join \ell_2 $ we get}
\TValGpcc{w_1}{\says{\ell_1}{τ}} \label{bse12} \\ 
\TValGpcc{w_2}{\says{\ell_1}{τ}} \label{bse13} \\ 
\TValGpcc{w_1'}{\says{\ell_2}{τ}} \label{bse14} \\ 
\TValGpcc{w_2'}{\says{\ell_2}{τ}} \label{bse15} \\
\intertext{from \ref{bse12},\ref{bse14},\ref{bse6}, \ref{bse7}, \ref{bse5} 
and SELECT rule we get}
\TValGpcc{\select{w_1}{w_1'}{\tau}}{\tau} \label{bse12} \\
\intertext{from \ref{bse13},\ref{bse15},\ref{bse6}, \ref{bse7}, \ref{bse5} and SELECT 
rule we get}
\TValGpcc{\select{w_2}{w_2'}{\tau}}{\tau} \label{bse13} 
%\intertext{from}
\end{align}
\begin{lemma}[Subject reduction(inter-host)]
If {\TValGpcS{\distcon{e}{c}{s}}{\tau}} and 
$\distcon{e}{c}{s} \Longrightarrow \distcon{e'}{c'}{s'}$ holds, then
{\TValGpcS{\distcon{e'}{c'}{s'}}{\tau}}.
%if $\rafjudge{\Pi}{c'}{pc}$. 
\end{lemma}
\textbf{Proof.}Induction over typing derivation of $e$.\\
\textbf{Case E-RUN}.
Given,
\begin{align}
{\distcon{E[\run{\hat{\tau}}{e}{c'}]}{c}{s}} \Longrightarrow {\distcon{\ret{e}{c}}{c'}{\stackapp{E[\expect{\hat{\tau}}]}{c}{s}}} \\
\TValGpcS{\distcon{E[\run{\hat{\tau}}{e}{c'}]}{c}{s}}{\tau} \label{sr1} \\
%\rafjudge{{\Pi}^{c'}}{c'}{pc}  \label{sr1.1} \\
\intertext{need to prove}
\TValGpcS{\distcon{\ret{e}{c}}{c'}{\stackapp{E[\expect{\hat{\tau}}]}{c}{s}}}{\tau} \label{sr2}
\intertext{Inverting \ref{sr1} we get}
\TValGpart{E[\run{\hat{\tau}}{e}{c'}]}{\tau'}{c'}\label{sr3} \\
\TValGpcS{s[\tau']}{\tau} \label{sr4} \\
\rafjudge{\Pi}{c}{pc} \label{sr5} \\
\stflowjudge*{pc}{pc'} \label{sr6} \\
\intertext{applying lemma CTX to \ref{sr3} we get} 
\TVal{{{\Pi}^{c}};\Gamma;pc';c}{\run{\hat{\tau}}{e}{c'}}{\hat{\tau}}\label{sr7} \\
%\TValP{\Gamma,x\ty \hat{\tau};pc';c}{E[x]}{\tau'[x\notin dom(\Gamma)]} \label{sr8} \\
\intertext{inverting \ref{sr7} we get}
\TVal{{{\Pi}^{c'}};\Gamma,\hat{\pc};c'}{e}{\hat{\tau}} \label{sr9} \\
\rafjudge{{\Pi}^{c}}{c}{pc'} \label{sr9.5} \\
\stflowjudge*{pc'}{\hat{pc}} \label{sr10} \\
\intertext{applying pc-reduction on \ref{sr9} with $\pc$ we get }
\TVal{{\Pi}^{c'};\Gamma,pc;c'}{e}{\hat{\tau}} \label{sr9.6} \\
\intertext{applying clearance in \ref{sr9.6}}
\rafjudge{{\Pi}^{c'}}{c'}{pc}  \label{sr9.7 we get} \\
\intertext{applying clearance lemma in \ref{sr9}}
\rafjudge{{\Pi}^{c'}}{c'}{\hat{pc}} \label{sr11} \\ 
\intertext{from \ref{sr6} and \ref{sr10} we get}
\stflowjudge*{pc}{\hat{pc}} \label{sr12} \\
\intertext{from \ref{sr11} and \ref{sr9} and \ref{sr11} and RET we get}
\TVal{{\Pi}^{c'};\Gamma,\hat{\pc};c'}{\ret{e}{c}}{\hat{\tau}} \label{sr13} \\
\intertext{from \ref{sr3} and \ref{sr7} and WAIT lemma we get}
\TValGpcc{E[\expect{\hat{\tau}}]}{\tau'}\label{sr13.5} \\
\intertext{from \ref{sr4}, \ref{sr5}, \ref{sr6},\ref{sr13.5} and TAIL rule we get}
\TValGpcS{\stackapp{E[\expect{\hat{\tau}}]}{c}{s[\hat{\tau}]}}{\tau} \label{sr14} \\
\intertext{from \ref{sr12},\ref{sr13},\ref{sr14} \ref{sr9.7} and HEAD rule we get}
\TValGpcS{\distcon{\ret{e}{c}}{c'}{\stackapp{E[\expect{\hat{\tau}}]}{c}{s}}}{\tau} \label{sr2}
\end{align}
\textbf{Case E-RETV}. Given, \\
\begin{align}
{\distcon{\ret{w}{c}}{c'}{\stackapp{E[\expect{\tau'}]}{c}{s}}} \Longrightarrow {\distcon{E[w]}{c}{s}} \\
\TValGpcS{\distcon{\ret{w}{c}}{c'}{\stackapp{E[\expect{\tau'}]}{c}{s}}}{\tau} \label{retv1} \\
\rafjudge{\Pi}{c}{pc}  \label{retv1.1} \\
\intertext{need to prove}
\TValGpcS{\distcon{E[w]}{c}{s}}{\tau} \label{retv2} \\
\intertext{inverting \ref{retv1} we get}
\TValP{\Gamma;pc';c'}{\ret{w}{c}}{\tau'} \label{retv3} \\
\TValGpcS{\stackapp{E[\expect{\tau'}]}{c}{s}[\tau']}{\tau} \label{retv4} \\
\stflowjudge*{pc}{pc'} \label{retv5} \\
\rafjudge{\Pi}{c'}{pc} \label{retv6} \\
\intertext{inverting \ref{retv3} we get} 
\TValP{\Gamma;pc';c'}{w}{\tau'} \label{retv7} \\ 
\rafjudge{\Pi}{c}{UB(\tau')} \label{retv7.5} \\
\rafjudge{\Pi}{c'}{pc'} \label{retv8} \\
\intertext{inverting \ref{retv4}}
\TValP{\Gamma;\hat{pc};c}{E[\expect{\tau'}]}{\hat{\tau}} \label{retv9} \\
\stflowjudge*{pc}{\hat{pc}} \label{retv10} \\ 
\rafjudge{\Pi}{c}{pc} \label{retv11} \\
\TValGpcS{s[\hat{\tau}]}{\tau} \label{retv12} \\
\intertext{applying clearance lemma on \ref{retv9}}
\rafjudge{\Pi}{c}{\hat{pc}} \label{retv13} \\
\intertext{from \ref{retv13}, \ref{retv7}, \ref{retv7.5} and ValuesHost lemma}
\TValP{\Gamma;\hat{pc};c}{w}{\tau'} \label{retv14} \\
\intertext{pc reduction in \ref{retv14}}
\TValP{\Gamma;{pc};c}{w}{\tau'} \label{retv14.1} \\
\intertext{clearance lemma on \ref{retv14.1} }
\rafjudge{{\Pi}^{c}}{c}{pc} \label{retv14.2} \\
\intertext{from \ref{retv14}, \ref{retv9} and WAITR lemma }
\TValP{\Gamma;\hat{pc};c}{E[w]}{\hat{\tau}} \label{retv15} \\  
\intertext{from \ref{retv15}, \ref{retv10}, \ref{retv12}, \ref{retv14.2} 
and HEAD rule we get}
\TValGpcS{\distcon{E[w]}{c}{s}}{\tau} 
\end{align}
\textbf{Case E-DSTEP.}\PM{generalize WAIT RWAIT lemma first.}
\PM{check if UB() condition is required in run}
\textbf{Case B-DSTEP.}
\textbf{Case B-RET.} Given,
\begin{align}
{\TValGpcw{\ret{\bracket{e_1}{e_2}}{c'}}{\says{c{^a}}{\tau}{\jmath({\tau})}}} \label{bret0} \\
\ret{\bracket{e_1}{e_2}}{c'} \stepsone \bracket{\ret{e_1}{c'}}{\ret{e_2}{c'}} 
\label{bret1} \\
\intertext{inverting \ref{bret0} we get,} 
\TValGpcw{\bracket{e_1}{e_2}}{\tau} \label{bret2} \\
\rafjudge{\Pi}{c'}{UB(\tau)} \label{bret2.2} \\
\rafjudge{\Pi}{c}{pc} \label{bret2.3} \\
\intertext{inverting \ref{bret2} we get,}
\TValGpcc{e_1}{\tau} \label{bret3} \\
\TValGpcc{e_2}{\tau} \label{bret4} \\
\stflowjudge*{H^{\pi}}{\tau} \label{bret5} \\
\rafjudge{\Pi}{c}{pc} \label{bret6} \\
\stflowjudge*{H^{pi} \join pc}{pc'} \label{bret7} \\
e_1 = v_1  \iff e_2 \ne v_2 \label{bret8} \\
\intertext{applying clearance lemma in \ref{bret3} we get,}
\rafjudge{\Pi}{c}{pc'} \label{bret9} \\
\intertext{from \ref{bret3}, \ref{bret9}, \ref{bret2.2} and RET rule we get}
\TValGpcc{\ret{e_1}{c'}}{\says{c^a}{\tau}{\jmath{(\tau)}}} \label{bret10} \\
\intertext{Similarly, from \ref{bret4}, \ref{bret9}, \ref{bret2.2} and RET rule we get,}
\ret{e_2}{c'}{\says{c^a}{\tau}{\jmath{(\tau)}}} \label{bret11} \\
\intertext{from \ref{bret8} we can say the following(\PM{or argue that 
any ret expression is not a value}),}
\ret{e_1}{c'} = v_1  \iff \ret{e_2}{c'} \ne v_2 \label{bret12}  \\
\intertext{from \ref{bret10},\ref{bret11},
\ref{bret12},\ref{bret2.3},\ref{bret5},\ref{bret7} and BRACKET rule we get,}
\TValGpcw{\bracket{\ret{e_1}{c'}}{\ret{e_2}{c'}}}{\says{c^a}{\tau}{\jmath{(\tau)}}}
\end{align}
%\textbf{Case B-RETV.}






\textbf{Example where propagation fails in RUN/RET.}\\
$\config*{\run{\tau}{(\assume{e_1}{e_2})}{c'}} \stepsone 
\configh*{\ret{(\assume{e_1}{e_2})}{c}} \stepsone^{*}
\configh*{\ret{(\assume{v}{e_2})}{c}} \stepsone
\configh*{\ret{(\where{e_2}{v})}{c}} \stepsone^{*}
\configh*{\ret{(\where{w}{v})}{c}}
$ \newline
Now we can step to \\
$\configh*{\where{(\ret{w}{c})}{v}}$ OR \\
$\config*{\where{w}{v}}$.\\ \\
First one propagates $\where{}{}$ term
and gets stuck as we dont have an evaluation rule for that
and the second one uses [E-RETV] rule 
and does not propagate the $\where{}{}$ term.

\section{Examples.}
In example 1, c runs the code that has authority as same as $a \vee b$
but needs a value that has integrity and confidentiality of $a \wedge b$.

\begin{figure*}
\begin{flushleft}
  \rulefiguresize
\begin{mathpar}
\func{\tau}{a \vee b}{\says{(a \wedge b)}{\tau}} \\
\lamc{x}{\tau}{a \vee b}\\
{\compEnd{\says{(a \wedge b)}{\tau}}{(\run{\says{a}{\tau}}{(\bind{y}{x}
{\returnv{a}{x}})}{a})}{(\run{\says{b}{\tau}}{(\bind{y}{x}
{\returnv{b}{x}})}{b})}}
\hfill
\end{mathpar}
\end{flushleft}
\caption{example 1}
\label{fig:example 1}
\end{figure*}
