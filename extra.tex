\newcommand{\var}[1]{\text{\texttt{#1}}}
\newcommand{\func}[1]{\text{\textsl{#1}}}
\newcounter{phase}[algorithm]
\newlength{\phaserulewidth}
\newcommand{\setphaserulewidth}{\setlength{\phaserulewidth}}
\newcommand{\phase}[1]{%
  \vspace{-1.25ex}
  % Top phase rule
  \Statex\leavevmode\llap{\rule{\dimexpr\labelwidth+\labelsep}{\phaserulewidth}}\rule{\linewidth}{\phaserulewidth}
  \Statex\strut\textit{$\triangleright$ #1 }% Phase text
  % Bottom phase rule
  \vspace{-1.25ex}\Statex\leavevmode\llap{\rule{\dimexpr\labelwidth+\labelsep}{\phaserulewidth}}\rule{\linewidth}{\phaserulewidth}}
\setphaserulewidth{.7pt}

\begin{algorithm}
  \caption{Voting Protocol}
  \label{alg:phoenix-voting}
  \begin{algorithmic}[1]
    \phase{\textbf{as a replica k}} \\
    When replica \texttt{j} is suspected\\
    \hskip 1em broadcast $\langle VOTE, j, getLatestDec(\texttt{DecLog})$, \O $\rangle_{\sigma_{k}}$\\
  \vspace{0.1cm}
    When replica \texttt{j} is detected with Proof \\
    \hskip 1em broadcast $\langle VOTE, j, getLatestDec(\texttt{DecLog}), Proof \rangle_{\sigma_{k}}$\\
  \vspace{0.1cm}
    When received m : $\langle VOTE, j, Cert, Proof \rangle_{\sigma_{i}}$ from
    replica \texttt{i}\\
    \hskip 1em \textbf{if} \texttt{i} in \texttt{Conf} $\land$ \texttt{m.j} in
    \texttt{Conf} $\land$ isValidCert(\texttt{m.Cert})\\
\end{algorithmic}
\end{algorithm}

First, let us 
\begin{figure}
  \begin{minted}[xleftmargin=0.6cm, linenos]{haskell}
availLarBal :: Choreo IO (Async Int @ "client")
availLarBal = do 
  bal1 <-
    bank1 `locally` \_ -> do putStrLn "Enter balance at bank1::"
      readLn :: IO Int

  bal2 <-
    bank2 `locally` \_ -> do putStrLn "Enter balance at bank2::"
      readLn :: IO Int
  
  bal1' <- (bank1, bal1) ~> client
  bal2' <- (bank2, bal2) ~> client
  availBal <- client `locally` \un ->  select (un bal1') (un bal2') 
  largeBal <- client `locally` \un ->  getLargest (un bal1') (un bal2')
  largestAvailBal <- client `locally` \un -> select (un largeBal) (un availBal) 

  client `locally` \un -> do 
    a <- wait $ un largestAvailBal
    putStrLn $ "largest available balance:" ++ show a
\end{minted}
\caption{\HasChor\ (fault-tolerant features added) implementation of largest available balance}
\end{figure}

% For hiding the leakages associated with SSE, different techniques have been
% used. ORAM based approaches are known to be most popular for hiding access pattern leakages. 
% %This is because, the server can correlate two ORAM access of the same data with negligible probability. 
% But ORAM based solutions are also very inefficient. 
% %because of the extra memory accesses. 
% Padding techniques are used to hide volumetric leakages. Again,it will only cause the client to read extra data on the server. Fully homomorphic encryption also incurs huge computational overhead. In a nutshell, all the mentioned %techniques offer security but at the same time make SSE operations time consuming. 