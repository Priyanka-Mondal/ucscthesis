\chapter{Introduction}
%\subsection{Distributed Systems and Decentralized Trust.}
In modern computing, distributed systems are ubiquitous, supporting various online services by integrating hosts across independent domains. 
%
However, building secure distributed systems is complex due to decentralized trust; i.e. participating hosts do not trust each other or any central entity. 
%
While hosts within a domain may have mutual trust, secure data flow between domains is necessary for meaningful computation and transactions (e.g. using one's credit card details on an e-commerce website for online shopping). 
%
Additionally, in this era of cloud computing and cloud storage, data must be securely stored to and accessed from remote databases outside the trusted domain. 
%
Maintaining inter-domain security and privacy involves addressing various issues, such as crash faults, Byzantine faults, network partitioning, and malicious servers. 
%
This complexity necessitates different mechanisms to ensure data and computation security in distributed systems. 
%
Quorum replication, information-flow control, and encryption are three widely used techniques in this regard. 
%
Throughout the chapters of this thesis, we will explore the design and implementation details of innovative methodologies based on these three techniques, aiming to make distributed systems trustless.
%
%
% First, this thesis presents a general purpose programming model called Flow Limited Authorization for Quorum Replication (FLAQR), that enables programmers to write secure decentralized applications, using quorum replication and information-flow-control. 
% %
% FLAQR uses the Flow-Limited Authorization model (FLAM) \cite{flam} to implement information flow control. 
% %
% FLAM algebra offers a unified model for reasoning about both information flow control and authorization. 
% %
% FLAQR incorporates availability policies into the FLAM algebra which only 
% reasoned about confidentiality and integrity policies. 
% %
% Another significant contribution of FLAQR is the introduction of secure \emph{fault-tolerant} language constructs to build quorum protocols.
% %
% FLAQR is the first language that addresses systems that compose multiple quorums or consider
% quorum participants with heterogeneous trust relationships.
% %
% Next, the thesis presents an extension of FLAQR, which has language constructs to support secret sharing as a form of declassification. 
% %
% The resulting language is called FLAQR$^+$. 
% %
% Next, the thesis presents FlameChor, a Haskell library called \FlameChor, that incorporates availability policies into \FLAM, and integrates fault-tolerance features and information-flow-control into choreographic programming.
% %
% Finally, we will discuss about Searchable Encryption (also referred as Structured Encryption and Encrypted Search), a technique that enables users to perform operations on encrypted data that is stored on untrusted remote databases securely. Particularly, we will present a novel technique that integrates forward and backward privacy with I/O efficiency for dynamic searchable encryption.



\section{Flow-Limited Authorization for quorum replication.}
\noindent This thesis presents a general purpose programming model called Flow Limited Authorization for Quorum Replication (FLAQR), that enables programmers to write secure decentralized applications, using quorum replication and information-flow-control. 
%
\paragraph{Information-flow-control.} Information-flow-control (or \textbf{IFC} in short) is a security model, which enables one to design security policies 
to describe the end-to-end behavior of the distributed system. 
%
Specifically, IFC provides a framework for specifying policies on the data that flows around in the distributed systems. 
%
The data is annotated with security labels. 
%
The data may flow from one host to another only if the security policy associated with the labels and the interacting hosts satisfy. 
%
IFC policies are of three kinds: confidentiality, integrity and availability policies. 
\begin{itemize}
\item A confidentiality policy ensures that secret data does not flow to public domain. 
\item An integrity policy ensures that a host does not accept data that is coming from an untrusted domain (i.e. domain that the host does not trust). 
\item An availability policy ensures that highly available data does not depend on data that is coming from low available domain. 
\end{itemize}

There have been a lot of work for ensuring confidentiality and integrity policies in ditributed systems \cite{jfabric,flac,jflac,deflate,steve06,myerspopl99,zznm01-award}. 
%
But, very less attention has been paid to availability policies. 
%
The only two works we are aware of that combined availability with IFC in the past are by Zheng and Myres \cite{qimp, avail}. 
%
Although \cite{qimp, avail} enforces availability with IFC, their notion of information flow labels are based on decentralized label model (DLM) \cite{myers-phdthesis,myers-phd-tr}. 
%
In that model, security labels and host principals are considered to be from two separate set of entities. 
%
Arden et. al. introduced Flow Limited Authorization Model(FLAM)\cite{flam,flamtr} 
in their CSF'15 paper. 
%
In this model security labels and host principals are considered to be elements of the same security lattice. 
%
The security policies in FLAM are represented as boolean expressions. 
%
FLAM talks about only confidentiality and integrity policies. 
%
One of the major contributions of FLAQR is incorporating availability policies into FLAM algebra. 
%We will discuss this in depth in Chapter \ref{ch:flaqr}.
%

\paragraph{Quorum replication.} Reasoning about availability is crucial for fault-tolerance. And, 
fault-tolerance is one of the main design goals of distributed systems. 
%
Quorum systems are a popular method for achieving fault tolerance in distributed applications by replicating data and computation across multiple hosts.
%
A quorum system is a collection of subsets of a set of hosts, with each subset referred to as a quorum. 
%
The quorums in the system must overlap sufficiently to ensure consistency.
%
Quorum replication operates under a failure model that specifies which hosts can fail. 
%
Specifically, it is defined by the maximum number of faulty hosts that can be tolerated in the system.
%
Quorum replication can be either homogeneous or heterogeneous. 
In homogeneous quorum replication systems, all participating hosts are equally trusted. 
%
In contrast, heterogeneous quorum replication systems may have some hosts that are trusted more than others.
%
The quorum system makes progress 
if a subset of hosts (i.e. one of the quorums) reach consensus 
by agreeing on their computed results. 
%
Thus, quorum systems can not only provide increased availability through replication, but also offer increased integrity through consensus. 
%
Although, quorum systems do not provide any confidentiality guarantees. 
%While IFC and quorum systems together can enforce all three (i.e. confidentiality, integrity and availability) security properties.

%-------------------
Writing quorum-based programs can be a painstaking task. 
%
There can be many repetitive checks for the parts of the code running on each individual quorum, making it verbose and hard to debug. 
%
Additionally, programmers can make logical mistakes, and unsafe data flows may occur while running the code. 
%
A programming language specifically designed to build fault-tolerant applications would be beneficial, 
allowing developers to write programs that are secure by construction.
%
 It is not as straightforward as it sounds, mainly due to the different inter and intra-domain 
 trust relationships among the hosts in a federated system.
 %
 FLAQR is the first language that lets you write programs for heterogeneous quorum systems.
%  %In Chapter \ref{ch:flaqr} we will discuss 
%  about Flow Limited Authorization for Quorum Replication (FLAQR), 
%  a general purpose programming framework for building distributed
% applications with heterogeneous quorum replication protocols.
FLAQR enforces end-to-end information security using FLAM algebra that is extended with availability policies.
FLAQR comes with a type-sytem that ensures any FLAQR program doea not violate their type-level specifications.
%
FLAQR is presented in detail in chapter \ref{ch:flaqr}. In chapter \ref{ch:flaqrplus}, an extended version of FLAQR is presented 
that supports supports secret sharing as a form of declassification.

\paragraph{Threat model of IFC based systems vs Distributed systems.}
 Different systems are built keeping different threat models in mind. 
 %
 Security guarantees of a system is proven with respect to its threat model. 
 %
 In an IFC based system the information flow labels are represented as points in a security lattice, and the underlying type system statically enforces IFC policies on data processed by the programs.
 %
 The hosts in IFC based systems is represented as principals. 
 %
 In Flow Limited Authorization Model\cite{flam} principals and security labels are treated equivalently, i.e. they are part of the same security lattice. 
 %
 The most powerful attacker in this system is represented as a conjunction of the atomic labels 
 and is a single point in the security lattice. 
 %
Distributed systems deal with malicious hosts. 
%
The security assumption underlying most fault-tolerant distributed protocols 
(such as Paxos\cite{paxos-techreport,paxos} and pBFT\cite{pbft}) is that there can only be a 
bounded number (say $f$) of faulty hosts.
%
To prove the security of distributed fault-tolerant programs annotated with IFC labels, the threat model of traditional IFC-based systems must be combined with the bounded fault assumptions of distributed systems.
%
To achieve this goal, the \emph{single most powerful} attacker model of IFC-based systems needs to be revised, and the most powerful attackers need to be represented as a set.
%
This new attacker model is discussed in detail in section \ref{sec:availAttacks} of chapter \ref{ch:flaqr}.
%
Special attention is paid to availability attackers, as this aspect has been missing in the literature from the perspective of the Flow Limited Authorization Model.
%We will discuss this attacker model in detail in \ref{sec:availAttacks}

%\paragraph{Security guarantees in terms of Noninterference.}
%\subsection{Flow-Limited Authorization for Secret-sharing}Chapter \ref{ch:flaqrplus}
\section{FlameChor: a Haskell library}
\noindent In the pursuit of implementing the fault-tolerant features of the FLAQR programming model, a Haskell library called FlameChor is developed. 
%
FlameChor is not a direct implementation of FLAQR; rather, it is a combination of several distinct features. 
%
FlameChor is implemented in few stages. 
%
The journey began with the enhancement of FLAME\cite{nmifc}—a Haskell implementation of FLAM, with availability policies, 
as Flame originally supported only confidentiality and integrity policies. 
%
The next step involved incorporating the IFC policies from the extended version of FLAME into HasChor\cite{haschor}, which is a functional choreographic programming framework. 
% 
Building on this, FLAQR-inspired fault-tolerant language constructs were added to HasChor. 
%
The result of these efforts is FlameChor, a Haskell library that merges the information-flow-control of FLAME with the choreographic programming framework offered by HasChor and FLAQR-based fault-tolerant language features.
%
Finally, to validate the development, examples such as a 2/3 majority quorum and leader-based consensus were implemented in FlameChor. 
%
In Chapter \ref{ch:flamechor}, the implementation of \FlameChor\ is discussed in detail.


\section{Privacy and Efficiency of Dynamic Searchable Encryption schemes.}
\noindent This thesis presents the first set of Dynamic Searchable Encryption(DSE) schemes that are Forward and Backward private and also achieves I/O-efficiency. 
\paragraph{Dynamic Searchable Encryption.}
The vast majority of distributed applications handle enormous amounts of data. 
%
In almost every application, clients store their data on remote servers, primarily because clients have limited memory (such as cell phones). 
%
Obviously, this data cannot be stored in plaintext because it often contains sensitive information (e.g., SSNs, medical records, bank details, personal emails, etc.). 
%
Therefore, clients first encrypt their data before storing it on the remote server. 
%
However, there is a problem with naively encrypting this data: the client cannot perform any computations on the encrypted data. %
If the client needs to access or update the data, it must download all of it locally, which is impractical. 
%
As a solution, Song et al.\cite{song2000practical} introduced symmetric searchable encryption, which provides a way to perform queries on encrypted data. 
%
Since then, a series of works have been done on searchable encryption \cite{kamara2017boolean,faber2015rich,Demertzis17,compress,demertzis2018practical,Demertzis16,faber2015rich,Kamara16}.


Searchable encryption is a special form of structured encryption, and also popularly known as encrypted search. 
%
In this thesis we will only consider symmetric key searchable encryption schemes (SSE). 
%
Searchable encryption enables clients to outsource their data to a remote server in ecrypted form while retaining the ability to access it without downloading it all. 
%
The main idea is that, given a set of encrypted documents, the client needs to be able to perform keyword searches on those documents.
%
To achieve this, the client first creates an inverted index from the documents. The index and the documents are then encrypted separately and uploaded to the server. 
%
The inverted index maps keywords to the document addresses (or identifiers) that contain them. 
%
During a search query, the client queries the index first, retrieves the document identifiers, and then retrieves the actual documents using these identifiers.
%
Searchable encryption can be of two kinds: static and dynamic. In static schemes, the set of searchable documents is fixed from the beginning. Once uploaded to the server, only search queries can be performed on them. %
In contrast, a dynamic searchable encryption (DSE) scheme allows for adding new documents, deleting old ones, and updating existing ones, in addition to performing searches.


\paragraph{Leakages and Forward \& Backward Privacy.} 
Ideally, there should not be any leakage in a searchable encryption scheme. However, to make a searchable encryption scheme practical, some leakages are tolerated. These leakages can be of different kinds.
%
The leakage that occurs during the initialization of the SE scheme is called the \emph{setup leakage}, or \emph{total leakage}, which includes the number of entries, the size of the search index, and the size of the documents. This leakage is unavoidable.

The leakage that occurs during client queries can be divided into two kinds: \emph{search-pattern} and \emph{access-pattern} leakages. Access pattern leakages are further divided into \emph{volume} and \emph{overlapping-pattern} leakages. Search pattern leakage reveals if two queries are for the same keyword. Volume pattern leakage reveals the size of the query result. Overlapping pattern leakage reveals if there are common documents in the results of any two queries. These leakages should be minimized as much as possible as prior works have shown that the aforementioned 
leakages can be exploited to perform query and database recovery attacks.
%
The most secure approach to defend against \emph{search} and \emph{overlapping} pattern leakages is to use Oblivious RAM (ORAM). ORAM hides memory access patterns so that the server cannot correlate accesses to the same memory location. On the other hand, a simplistic approach for mitigating \emph{volume-pattern} leakages is to pad every result with extra (dummy) items.
%

Dynamic symmetric searchable encryption (DSE) leaks additional information due to insertions and deletions. There are two security notions for dynamic SSE schemes: \emph{forward} and \emph{backward} privacy. Forward privacy ensures that a new document insertion does not reveal the keywords this document contains that have been queried before. Lack of forward privacy makes the scheme susceptible to file-injection attacks\cite{yqbu}. Backward privacy ensures that a search query does not reveal anything about entries that have already been deleted. For example, if a document \(d\) containing keyword \(w\) is deleted before a search for \(w\), then the result of this search does not reveal anything about \(d\). To comply with privacy regulation laws like GDPR, CCPA, HIPAA etc. ensuring backward privacy is important.
%
Backward privacy is categorized into three types:
(a) Type-I reveals only the identifiers of documents currently containing the searched keyword and the timestamp of when the documents were inserted,
(b) Type-II additionally reveals the timestamps and types of all prior updates for the searched keywords (i.e., insertions and deletions),
(c) Type-III also reveals which insertion each deletion canceled.

% \paragraph{Client Storage}
% Minimizing the client storage is another important aspect of designing practical SSE schemes. This is because in real world applications the client storage is very limited (e.g. in cell phones). Also, sending the data to the server as much as possible, makes it flexible for the user to switch between devices and use the same data over the cloud. Vast majority of the efficient DSE schemes maintain one 
% counter for each unique keyword. This counter is either stored locally at the
% client or stored in encrypted form at the server and accessed obliviously. 
% ORAM based solutions need to maintain a position map 
% in the server, although we can get rid of the position map with recursive ORAMs.
% Demertzis et. al. proposed three DSE schemes in their NDSS'20 \cite{SDa} paper 
% with constant client storage. In \cite{mishraoblix} they use SGX to store the client's internal state at the server.

% \paragraph{Threat model in searchable encryption.}
% Attacks in distributed systems can be of different types. 
% The type of the attack depends on multiple factors. 
% It depends on the adversarial model in which they work, 
% the information the adversary have, 
% or the inputs the adversary can produce.  
% Query-recovery attacks recover the queries made by the client, 
% whereas data-recovery attacks recover client's data.  
% For snapshot attacks, the adversary needs a hold over the document
% collection, whereas for persistent attacks it needs to know the query
% tokens as well. Sampled-data attacks need a sample from 
% the distribution that is very close to the distribution 
% of the original client data; known-data attacks require knowledge of a subset of the client's data collection, and known-query attacks require
% knowledge of a subset of the client queries. 
% Attackers can be of two types: passive and active.
% Passive attackers only observe the queries and memory 
% access patterns at the server and based on the 
% observations they try to infer the queries and client's data
% accessed by the queries.
% Active attackers can also inject data into the database.


 
\paragraph{I/O Efficiency.}
To make searchable encryption schemes useful for real-world applications, attention needs to be paid to their efficiency.
%
In modern cryptography, really fast symmetric cryptographic primitives are available, and because of this, the cost of memory accesses, specifically the number of I/O operations, becomes the performance bottleneck.
%
The fact of the matter is that the performance of SSE operations can be improved if the number of I/O operations is minimized.
%
There are some I/O metrics to measure I/O efficiency. The metrics relevant when the data is stored on HDDs are \emph{locality} and \emph{read efficiency}.
%
\emph{Locality} refers to the number of contiguous memory locations the server accesses to perform a client query. 
%
To minimize the number of random disk seeks, careful consideration is needed for how entries pertaining to same result are stored on the server. 
%
Storing the related entries in contiguous memory locations is insecure, as it violates forward privacy.
%
Storing each entry at a random location is ensures forward privacy but suffers from poor locality. 
%
However, \emph{locality} alone is not a meaningful metric since the entire database can be read with optimal locality.
 %
 Therefore, \emph{read efficiency} must also be considered. 
 %
 Read efficiency is the ratio of the amount of data actually read by the server to the amount of data that corresponds to the query result.
 %
 Ideally, both locality and read efficiency should be minimized.
%
For SSDs, the I/O efficiency metric is called \emph{page efficiency}. 
Page efficiency refers to the ratio of the number of pages read to fetch the encrypted data to the number of pages that would have been read if the data were stored in plaintext.

There has been a line of work \cite{onechoice,LocalLayeredSSECrypto22,Demertzis17,locCrypto18} 
focused on making searchable encryption schemes I/O-efficient. Another line of work 
\cite{Mitra,ShiNDSS14, RBCC16} focuses on 
making DSE schemes Forward and Backward private.
Although some of the previous work \cite{LocalLayeredSSECrypto22,RBCC} claim that 
forward privacy and I/O efficiency cannot be achieved simultaneously, chapter \ref{ch:iodse} of this thesis presents schemes that demonstrates these two can indeed be achieved together.

%we will discuss the first dynamic searchable encryption schemes that are locality-aware, and at the same time ensure forward and backward privacy. %

%Cash and Tessaro \cite{cash2014locality}~ proved that any secure searchable encryption scheme with both optimal locality $O(1)$ and optimal read efficiency $O(1)$ requires $\omega(N)$ space.

%  \paragraph{System-wide view.}
%  A common feature in the majority of SSE schemes is that they consider leakages 
%  only from the encrypted search index. The same kind of leakage can take place during the document retrieval phase. Hence the existing leakage-abuse attacks based on co-occurrence patterns or volumes can also be extended to document level. Recently authors of \cite{precSwiSSE}~presented a query recovery attack from the leakages that happen during the document retrieval phase. 
%  It is tricky to use the leakage mitigation techniques that work for
% the encrypted search index for documents as well. The main problem is the document sizes: they can be huge and each one can be of different size. In Chapter 
% \ref{ch:systemWide}, we will discuss three DSE schemes that take system-wide leakage into account. 

% %\paragraph{B-tree Path-ORAM}
 
 
 
%  %\paragraph{security guarantees}
 
 

% \paragraph{Organization of the chapters.}
% Chapter \ref{ch:flaqr} talks about applying consensus and replication
% securely with "Flow Limited Authorization for quorum replication" (FLAQR),
% which is a core calculus for building distributed
% applications with heterogeneous quorum replication protocols.
% First we will see why availability is crucial in distributed systems,
% and why it is hard to enforce in section \ref{sec:Intro}. 
% We will see how quorum replication can help 
% to ensure availability. Then syntax, semantics, and type system 
% of FLAQR is shown in sections \ref{sec:FLAQRprimitives} and \ref{sec:types}, which helps us to enforce end-to-end information security.
% Next we will see noninterference theorems that
% characterize FLAQR’s confidentiality, integrity, and availability
% in the presence of consensus, replication, and failures in sections 
% \ref{sec:congIntNI}-\ref{sec:availNI}. Section \ref{blameproofs} 
% presents a liveness theorem for the class of majority quorum protocols
% under a bounded number of faults. 
% Chapter \ref{ch:iodse} presents the first dynamic searchable encryption schemes that are locality aware, forward and backward private. Section \ref{subsec:SDa} talks about instantiation of the \SDa scheme from 
% \cite{SDa} with locality aware schemes like One-Choice allocation, Two-choice allocation and $N\log N$ schemes from \cite{onechoice}, and then their search performance is compared with the original \SDa construction that was instantiated with \PiBas. But \SDa has amortized updates. Section \ref{sec:deamortized} shows why it can be difficult to maintain forward privacy and locality of the de-amortized scheme at the same time.

