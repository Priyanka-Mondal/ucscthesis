\section{Flow limited authorization for quorum replication}
\label{sec:fullrules}
\label{appendix}
\label{sec:ubrules}
\if 0
\begin{figure}[h]
  \small
  \[
    \begin{array}{rcl}
      \multicolumn{3}{l}{ \Pi \in {\{"c","i","a"\}}  \text{  (projections)}} \\
      \multicolumn{3}{l}{ n \in \N \text{  (primitive principals)}} \\
      \multicolumn{3}{l}{ x \in \mathcal{V} \text{  (variable names)}} \\
      \\
      p,\ell,\pc &::=&  n \sep \top \sep \bot \sep p^{\Pi} \sep p \wedge p\sep p \vee p \\[0.4em]
                   &\sep & p \sqcup p \sep p \sqcap p \sep \selor{p}{p} \sep \comor{p}{p} \\[0.4em]
      \tau &::=& \voidtype \sep X \sep \sumtype{\tau}{\tau} \sep \prodtype{\tau}{\tau} \\[0.4em]
         & \sep & \func{\tau}{\pc}{\tau} \sep \tfunc{X}{\pc}{\tau} \sep \says{\ell}{\tau}  \\[0.4em]
      v &::=& \void \sep \returnv{\ell}{v} \sep \injia{\sumtype{\tau}{\tau}}{v} \sep \paira{v}{v}{\tau} \\[0.4em]
        & \sep & \lamc{x}{\tau}{\pc}{e} \sep \tlam{X}{\pc}{e}  \\[0.8em]
      f &::=& v \sep  \faila{\tau} \\[0.8em]
      e &::=& f \sep x \sep e~e \sep e~\tau \sep \return{\ell}{e} \sep 
             \paira{e}{e}{\tau} \\[0.4em]
        & \sep & \proji{e} \sep \injia{\sumtype{\tau}{\tau}}{e} \sep \bind{x}{e}{e}\\[0.4em]
        & \sep & \casexpan{e}{x}{e}{e}{\tau} \\[0.4em]
        & \sep & \runa{\tau}{e}{p}  \sep \ret{e}{p}\\[0.4em]
        & \sep & \selecta{e}{e}{\tau} \sep \comparea{\tau}{e}{e} \sep \expecta{\tau}
    \end{array}
  \]
  \caption{Type annotated FLAQR Syntax (Full version).}
  \label{fig:Annotatedsyntax}
\end{figure}
\fi

\vspace{-1em}
\begin{figure}[h]
{\small
\begin{flalign*}
   \UB{\func{\tau_1}{pc}{\tau_2}} &= \UB{\tau_1} \join pc \join \UB{\tau_2} \\
   \UB{\tfunc{X}{pc}{\tau}} &= pc \join \UB{\tau} \\
   \UB{\says{\ell}{\tau}} &= \UB{\tau} \\
   \UB{\sumtype{\tau_1}{\tau_2}} &= \UB{\tau_1} \join \UB{\tau_2} \\
   \UB{\prodtype{\tau_1}{\tau_2}} &= \UB{\tau_1} \join \UB{\tau_2} \\
   \UB{\voidtype} &= \bot
\end{flalign*}
}
\vspace{-1em}
\caption{Clearance function}
\label{fig:UBFunction}
\end{figure}
\vspace{-1em}
\begin{figure}[h]
 {\small
  \begin{mathpar}
\erule{E-AppFailL}{}{{\lamc{x}{\tau}{pc}{\faila{\func{\tau}{pc}{\tau'}}}}~e}{\faila{\tau'}}\\

\erule{E-TAppFail}{}{\faila{\tfunc{X}{pc}{\tau}}~\tau'}{{\faila{\subst{\tau}{X}{\tau'}}}}


\erule{E-CaseFail}{}{\casexpan{\faila{\tau'}}{x}{e_1}{e_2}{}
{}}{\faila{\tau}}

\erule{E-PairFailL}{}{\paira{\faila{\tau_1}}{f_2}{\prodtype{\tau_1}{\tau_2}}}{\faila{\prodtype{\tau_1}{\tau_2}}}

\erule{E-PairFailR}{}{\paira{f_1}{\faila{\tau_2}}{\prodtype{\tau_1}{\tau_2}}}{\faila{\prodtype{\tau_1}{\tau_2}}} \hfill
\end{mathpar}
}
\vspace{-1em}
\caption{ Remaining cases for propagation of $\fail{}$ terms. }
\label{fig:Fullfailprop}
\end{figure}
\vspace{-1em}
\begin{figure}[h]
{\small
\begin{align*}
\mathscr{C}(\voidtype)                                       &=  \voidtype \\
\mathscr{C}(\sumtype{\tau_1}{\tau_2})                        &=  \sumtype{\mathscr{C}(\tau_1)}{\mathscr{C}(\tau_2)}\\
\mathscr{C}(\prodtype{\tau_1}{\tau_2})                       &=  \prodtype{\mathscr{C}(\tau_1)}{\mathscr{C}(\tau_2)}\\
\mathscr{C}(\func{\tau_1}{pc}{\tau_2})                       &=  \func{\mathscr{C}({\tau_1})}{pc}{\mathscr{C}({\tau_2})}\\
\mathscr{C}(\tlam{X}{pc}{\tau})                              &= \tlam{X}{pc}{\mathscr{C}(\tau)}\\
\mathscr{C}(\says{(\selor{\ell_1}{\ell_2})}{\tau})           &=  \says{(\ell_1 \vee \ell_2)}{\mathscr{C}{(\tau)}}\\
\mathscr{C}(\says{(\comor{\ell_1}{\ell_2})}{\mathscr{\tau}}) &=  \says{(\ell_1 \wedge \ell_2)}{\mathscr{C}{(\tau)}}\\
 \textsf{(otherwise)~} \mathscr{C}(\says{\ell}{\tau})                               &=  \says{\ell}{\mathscr{C}(\tau)} 
\end{align*}
}
\label{fig:Cfunction}
\vspace{-1em}
\caption{$\mathscr{C}$ function on types.}
\end{figure}

\begin{figure}
{\small
%\begin{flalign*}
%& \boxed{\readjudge{\Pi}{p}{\tau}} &
%\end{flalign*}
\begin{mathpar}
\Rule{R-Unit}
     {}
     {\readjudge{\Pi}{p}{\void}}
\Rule{R-TFun}
           {
            \readjudge{\Pi}{p}{\tau}
           }
           {\readjudge{\Pi}{p}{\tfunc{X}{pc}{\tau}}}

\Rule{R-Sum}
     {\readjudge{\Pi}{p}{\tau_1} \\\\
      \readjudge{\Pi}{p}{\tau_2}
     }
     {\readjudge{\Pi}{p}{\sumtype{\tau_1}{\tau_2}}}

\Rule{R-Prod}
     {\readjudge{\Pi}{p}{\tau_1} \\\\
      \readjudge{\Pi}{p}{\tau_2}
     }
     {\readjudge{\Pi}{p}{\prodtype{\tau_1}{\tau_2}}}

\Rule{R-Lbl}
        {\rafjudge{\delegcontext}{p^{\confid}}{\ell^{\confid}} \\\\
         \readjudge{\Pi}{p}{\tau} 
        }
        {\readjudge{\delegcontext}{p}{\says{\ell}{\tau}}}

\Rule{R-Fun}
           {
            \readjudge{\delegcontext}{p}{\tau_1} \\\\
            \readjudge{\delegcontext}{p}{\tau_2}
           }
          {\readjudge{\Pi}{p}{\func{\tau_1}{\pc}{\tau_2}}}
\end{mathpar}
}
\caption{Reads judgments.}
\label{fig:readjudgment}
\end{figure}
\begin{figure}
  {\small
\begin{flalign*}
\last(x,y,&\blame,\ell_1,\ell_2) =~ \MATCH~ (x,y)~ \WITH \\
|~& (\returnv{\ell}{v_1},\returnv{\ell}{v_2}) =  \\
&~ \IF~ \entails{\blame}{\inF{\ell_1}{\FN}}~ \THEN~ \blame \\
&~\ELSE~\IF~ \entails{\blame}{\inF{\ell_2}{\FN}}~\THEN~ \blame\\
&~\ELSE~\IF~ \entails{\blame}{\inF{\ell}{\FN}}~\THEN~ \blame\\
&~\ELSE ~\last(v_1,v_2,\blame,\ell,\ell) \\
|~& (\return{\ell}{e_1},\return{\ell}{e_2}) = \last(e_1,e_2,\blame,\ell_1,\ell_2) \\ 
|~& (\injia{\tau}{e_1},\injia{\tau}{e_2}) = \last(e_1,e_2,\blame,\ell_1,\ell_2) \\
|~& (\paira{e_{11}}{e_{12}}{\tau},\paira{e_{21}}{e_{22}}{\tau}) =  \\
  &~ \last(e_{11},e_{21},(\last(e_{12},e_{22},\blame,\ell_1,\ell_2)),\ell_1,\ell_2) ~ \\
|~& (\runa{\tau}{e_1}{p},\runa{\tau}{e_2}{p}) = \last(e_1,e_2,\blame,\ell_1,\ell_2) \\
|~& (\selecta{e_1}{e_2}{\tau},\selecta{e_1'}{e_2'}{\tau}) = \\
  &~\last(e_1,e_1', (\last(e_2,e_2',\blame,\ell_1,\ell_2)),\ell_1,\ell_2) \\ 
|~& (\comparea{\tau}{e_1}{e_2},\comparea{\tau}{e_1'}{e_2'}) = \\
  &~ \last(e_1,e_1', (\last(e_2,e_2',\blame,\ell_1,\ell_2)),\ell_1,\ell_2) \\
|~& (\lamc{x}{\tau}{pc}{e_1},\lamc{x}{\tau}{pc}{e_2}) = \last(e_1,e_2,\blame,\ell_1,\ell_2) \\
|~& (\tlam{X}{pc}{e_1},\tlam{X}{pc}{e_2}) = \last(e_1,e_2,\blame,\ell_1,\ell_2) \\
|~& (\proji{e_1},\proji{e_2}) = \last(e_1,e_2,\blame,\ell_1,\ell_2) \\ 
|~& (\bind{x_1}{e_1}{e_1'},\bind{x_2}{e_2}{e_2'}) = \\
  &~ \last(e_1,e_2,\last(e_1',e_2',\blame,\ell_1,\ell_2) ,\blame,\ell_1,\ell_2) \\
|~& (\casexpan{e_1}{z}{e_2}{e_3}{\tau}, \\
  &~~~ \casexpan{e_1'}{z}{e_2'}{e_3'}{\tau}) = \\
  &~~~~~last(e_1,e_1',\last(e_2,e_2',\last(e_3,e_3',\blame,\ell_1,\ell_2),\ell_1,\ell_2),\ell_1,\ell_2)\\
|~& (f_1,f_2) = \\
  &~ \IF~ f_1 = f_2 \THEN ~ \blame \\
  &~ \ELSE ~\IF~ \entails{\blame}{\inF{\ell_1}{\FN}}~ \THEN~ \blame \\
  &~ \ELSE ~\IF~ \entails{\blame}{\inF{\ell_2}{\FN}} ~\THEN~ \blame~ \\
  &~ \ELSE ~ "DNF"(\inF{\ell_1}{\FN} \AND \blame) \OR "DNF"(\inF{\ell_2}{\FN} \AND \blame)
\end{flalign*}
\begin{flalign*}
&"DNF"(\inF{\ell}{\FN} \AND \blame)~ =>~ \MATCH~ \blame~ \WITH &\\
&~~~~~~~~~  |~ \FN=\emptyset => \inF{\ell}{\FN} &\\
&~~~~~~~~~  |~ \inF{\ell'}{\FN} => \inF{\ell}{\FN} \AND \inF{\ell'}{\FN} &\\
&~~~~~~~~~  |~ \blame_1 \OR \blame_2 => "DNF"(\inF{\ell}{\FN} \AND \blame_1) \OR "DNF"(\inF{\ell}{\FN} \AND \blame_2) &\\
&~~~~~~~~~   |~ \blame_1 \AND \blame_2 => \blame_1 \AND \blame_2  \AND \inF{\ell}{\FN} &\\
\end{flalign*}
}
\caption{Function to construct blame constraint $\blame$.}
\label{fig:Blameconst}
\end{figure}

\begin{figure}
%\Rule{PartandL}
%\begin{figure*}
{\small
\hfill
  \begin{mathpar}
\Rule{PAndL}
{\rafjudge{\Pi}{p_i}{p} \\\\
  k \in \{1,2\}
 %\rafjudge{\Pi}{p_2}{p}
}
{\rafjudge{\Pi}{\comor{p_1}{p_2}}{p}}

\Rule{PAndR}
{\rafjudge{\Pi}{p}{p_1}\\\\
 \rafjudge{\Pi}{p}{p_2}}
{\rafjudge{\Pi}{p}{\comor{p_1}{p_2}}}

\Rule{AndPAnd}
{}
{\rafjudge{\Pi}{p \wedge q}{\comor{p}{q}}}

\Rule{PAndPOr}
{}
{\rafjudge{\Pi}{\comor{p}{q}}{\selor{p}{q}}}

\Rule{ProjPAndL}
{}
{\rafjudge{\Pi}{\comor{p^{\pi}}{q^{\pi}}}{(\comor{p}{q})^{\pi}}}

\Rule{ProjPAndR}
{}
{\rafjudge{\Pi}{(\comor{p}{q})^{\pi}}{\comor{p^{\pi}}{q^{\pi}}}}

\Rule{ProjPOrL}
{}
{\rafjudge{\Pi}{\selor{p^{\pi}}{q^{\pi}}}{(\selor{p}{q})^{\pi}}}

\Rule{ProjPOrR}
{}
{\rafjudge{\Pi}{(\selor{p}{q})^{\pi}}{\selor{p^{\pi}}{q^{\pi}}}}

\Rule{POrOr}
{}
{\rafjudge{\Pi}{\selor{p}{q}}{p \vee q}}

\\\\

\Rule{AndDistPOrR}{}
{\rafjudge{\Pi}{p \wedge (\selor{q}{r})}
{\selor{(p \wedge q)}{(p \wedge r)}}}

\Rule{POrDistAndR}{}
{\rafjudge{\Pi}{\selor{p}{(q \wedge r)}}
{(\selor{p}{q}) \wedge (\selor{p}{r})}}

\Rule{AndDistPOrL}{}
{\rafjudge{\Pi}
{\selor{(p \wedge q)}{(p \wedge r)}}
{p \wedge (\selor{q}{r})}}

\Rule{{POrDistAndL}}{}
{\rafjudge{\Pi}
{(\selor{p}{q}) \wedge (\selor{p}{r})}
{\selor{p}{(q \wedge r)}}}

\Rule{{OrDistPOrR}}{}
{\rafjudge{\Pi}{p \vee (\selor{q}{r})}
{\selor{(p \vee q)}{(p \vee r)}}}

\Rule{{OrDistPOrL}}{}
{\rafjudge{\Pi}
{\selor{(p \vee q)}{(p \vee r)}}{p \vee (\selor{q}{r})}}

\Rule{{POrDistOrR}}{}
{\rafjudge{\Pi}{\selor{p}{(q \vee r)}}
{(\selor{p}{q}) \vee (\selor{p}{r})}}

\Rule{{POrDistOrL}}{}
{\rafjudge{\Pi}
{(\selor{p}{q}) \vee (\selor{p}{r})}{\selor{p}{(q \vee r)}}}

\\\\

\Rule{AndDistPAndR}{}
{\rafjudge{\Pi}{p \wedge (\comand{q}{r})}
{\comand{(p \wedge q)}{(p \wedge r)}}}

\Rule{PAndDistAndR}{}
{\rafjudge{\Pi}{\comand{p}{(q \wedge r)}}
{(\comand{p}{q}) \wedge (\comand{p}{r})}}

\Rule{AndDistPAndL}{}
{\rafjudge{\Pi}
{\comand{(p \wedge q)}{(p \wedge r)}}
{p \wedge (\comand{q}{r})}}

\Rule{{PAndDistAndL}}{}
{\rafjudge{\Pi}
{(\comand{p}{q}) \wedge (\comand{p}{r})}
{\comand{p}{(q \wedge r)}}}

\Rule{{OrDistPAndR}}{}
{\rafjudge{\Pi}{p \vee (\comand{q}{r})}
{\comand{(p \vee q)}{(p \vee r)}}}

\Rule{{OrDistPAndL}}{}
{\rafjudge{\Pi}
{\comand{(p \vee q)}{(p \vee r)}}{p \vee (\comand{q}{r})}}

\Rule{{PAndDistOrR}}{}
{\rafjudge{\Pi}{\comand{p}{(q \vee r)}}
{(\comand{p}{q}) \vee (\comand{p}{r})}}

\Rule{{PAndDistOrL}}{}
{\rafjudge{\Pi}
{(\comand{p}{q}) \vee (\comand{p}{r})}{\comand{p}{(q \vee r)}}}
\hfill
\end{mathpar}
\label{fig:disjorandactsfor}
}
\caption{FLAQR Partial conjunction and disjunction acts-for rules.}
\label{fig:partialactsforfull}
\end{figure}

\begin{figure}
\begin{subfigure}{0.5\textwidth}
{\small
\begin{mathpar}
    \Rule{Bracket}                                                                                 {
           \rflowjudge{\delegcontext}{(H^\pi \sqcup \pc)}{{\pc'}} \\
           e_1 = v_1  \iff e_2 \ne v_2 \\\\                        
           \TValP{\Gamma;\pc';c}{e_1}{\tau} \\
           \TValP{\Gamma;\pc';c}{e_2}{\tau} \\\\
           \protjudge{\delegcontext}{H^{\pi}}{\cfun{\tau}}\\
           \rafjudge{\Pi}{c}{pc} %\\ \pi \in \{"i","c"\}
           }
         {\TValGpcw{\bracket{e_1}{e_2}}{\tau}}

  \Rule{Bracket-Values}                                                                            {
           \TValGpcw{v_1}{\tau} \\                               
           \TValGpcw{v_2}{\tau} \\\\  
           \protjudge{\delegcontext}{H^\pi}{\cfun{\tau}} \\
          \rafjudge{\Pi}{c}{pc}
          }
         {\TValGpcw{\bracket{v_1}{v_2}}{\tau}}


  \Rule{BullR}
  {\TValP{\Gamma;\pc;c}{e}{\tau} \\
   %\pi \in \{"i","c"\}
  }
  {\TValP{\Gamma;\pc;c}{\bracket{e}{\bullet}}{\tau}}
  \Rule{BullL}
  {\TValP{\Gamma;\pc;c}{e}{\tau} \\
   %\pi \in \{"i","c"\}
  }
  {\TValP{\Gamma;\pc;c}{\bracket{\bullet}{e}}{\tau}}

  \Rule{Bracket-Fail-L}
  {\TValP{\Gamma;\pc;c}{e}{\tau} \\
   %\pi \in \{"i","c"\}
  }
  {\TValP{\Gamma;\pc;c}{\bracket{e}{\faila{\tau}}}{\tau}}

  \Rule{Bracket-Fail-R}
  {\TValP{\Gamma;\pc;c}{e}{\tau} \\
  % \pi \in \{"i","c"\}
  }
  {\TValP{\Gamma;\pc;c}{\bracket{\faila{\tau}}{e}}{\tau}}
  
  \Rule{Bracket-Fail-A}
  {\TValP{\Gamma;\pc;c}{e_i}{\tau} \\
   e_i \not= \faila{\tau} \\
   \pi = "a"
  }
  {\TValP{\Gamma;\pc;c}{\bracket{e_1}{e_2}}{\tau}}
  
   
  \Rule{Bracket-Same}
  {\TValP{\Gamma;\pc;c}{v}{\tau} \\
  }
  {\TValP{\Gamma;\pc;c}{\bracket{v}{v}}{\tau}}

 \end{mathpar}
}
\caption{Typing rules for bracketed expressions.}
\label{fig:bracketTypes}

\end{subfigure}
\begin{subfigure}{0.5\textwidth}
{\small
\begin{mathpar}
\Rule{Bracket-Stack}
        {
          \TValGpcc{e}{\tau'}\\
          \drflowjudge{\Pi}{pc}{pc'} \\\\
          \forall i \in \{1,2\}.\TValGpcS{{s_i}}{[\tau']\tau}
        }
        {\TValGpcS{\distcon{e}{c}{\bracket{s_1}{s_2}}}{\tau}}

\Rule{Bracket-Head}
        {
          \TValGpcc{\bracket{e_1}{e_2}}{\tau'}\\
          \drflowjudge{\Pi}{pc}{pc'} \\\\
          \TValGpcS{s}{[\tau']\tau}
        }
        {\TValGpcS{\distcon{\bracket{e_1}{e_2}}{c}{s}}{\tau}}
%   \Rule{Bracket-Empty}
%        {}
%        {\TValGpcS{\bracket{empty}{empty}[\tau]}{\tau}}
\end{mathpar}
}
\caption{Typing rules for bracketed configuration stack.}
\label{fig:distbracketTypes}
\end{subfigure}
\caption{Bracketed typing rules.}
\end{figure}

\begin{figure*}
\small
\begin{mathpar}
\berule*{B-CompareCommon}
{\outproj{\comparea{\says{\txcmp{\ell_1}{\ell_2}}{\tau}}
{\bracket{f_{11}}{f_{12}}}
{\bracket{f_{21}}{f_{22}}}}{i}
\stepsone f_i \quad \quad \forall i \in \{1,2\}}
{\comparea{\says{\txcmp{\ell_1}{\ell_2}}{\tau}}{\bracket{f_{11}}{f_{12}}}{\bracket{f_{21}}{f_{22}}}}
{\bracket{f_1}{f_2}}
{}

\berule*{B-CompareCommonRight}
{\outproj{\comparea{\says{\txcmp{\ell_1}{\ell_2}}{\tau}}
{\bracket{f_{11}}{f_{12}}}
{f}}{i}
\stepsone f_i \quad \quad \forall i \in \{1,2\}}
{\comparea{\says{\txcmp{\ell_1}{\ell_2}}{\tau}}{\bracket{f_{11}}{f_{12}}}{f}}
{\bracket{f_1}{f_2}}
{}


\berule*{B-SelectCommon}
{\outproj{\selecta{\bracket{f_{11}}{f_{12}}}{\bracket{f_{21}}{f_{22}}}{\says{\txsel{\ell_1}{\ell_2}}{\tau}}}{i} \stepsone f_i
\quad \quad \forall i \in \{1,2\}}
{\selecta{\bracket{f_{11}}{f_{12}}}{\bracket{f_{21}}{f_{22}}}{\says{\txsel{\ell_1}{\ell_2}}{\tau}}}{\bracket{f_1}{f_2}}
{}

\berule*{B-SelectCommonLeft}
{\outproj{\selecta
{\bracket{f_{11}}{f_{12}}}
{f}{\says{\txsel{\ell_1}{\ell_2}}{\tau}}}{i}
\stepsone f_i \quad \quad \forall i \in \{1,2\}}
{\selecta{\bracket{f_{11}}{f_{12}}}{f}{\says{\txsel{\ell_1}{\ell_2}}{\tau}}}
{\bracket{f_1}{f_2}}
{}


\berule*{B-Fail1}{}
{\return{\ell}{\bracket{v}{\faila{\tau}}}}
{\bracket{\returnv{\ell}{v}}{\faila{\says{\ell}{\tau}}}}
{}

\berule*{B-Fail2}{}
{\return{\ell}{\bracket{\faila{\tau}}{v}}}
{\bracket{\faila{\says{\ell}{\tau}}}{\returnv{\ell}{v}}}
{}

\berule*{B-Fail}{}
{\return{\ell}{\bracket{\faila{\tau}}{\faila{\tau}}}}
{\faila{\tau}}
{}

\derule{B-RunLeft}{}{\distcon{\bracket{E[\runa{\tau}{e_1}{c'}]}{e_2}}
{c}{s}}{\distcon{\bracket{\ret{e_1}{c}}{\bullet}}{c'}
{\stackapp{\bracket{E[\expecta{\tau}]}
{e_2}}{c}{s}}}

\derule{B-RetRight}{
 f' = {\begin{cases}
     \returnv{\ell}{v} &\mbox{ if } f = v \\
     \faila{\says{\ell}{\tau}} &\mbox{ if } f = \faila{\tau}
      \end{cases}
     }}
 {\distcon{\bracket{\bullet}{\ret{f}{c}}}{c'}
{\stackapp{\bracket{e_2}{E[\expecta{\says{\ell}{\tau}}]}}
{c}{s}}}{\distcon{\bracket{e_2}{E[f']}}{c}{s}}

\derule{B-RetV}
{f_i' = {\begin{cases}
          \returnv{\ell}{v} &\mbox{ if } f_i = v \\
          \faila{\says{\ell}{\tau}} &\mbox{ if } f_i = \faila{\tau}
         \end{cases}
}}
{\distcon{\ret{\bracket{f_1}{f_2}}{c}}{c'}
{\stackapp{E[\expecta{\says{\ell}{\tau}}]}{c}{s}}}
{\distcon{E[\bracket{f_1'}{f_2'}]}{c}{s}}
{}
 \end{mathpar}
%\end{subfigure}
\caption{Selected bracketed Evaluation Rules.}
\label{fig:brackets}
\end{figure*}

\begin{figure*}
  {\small
\[
\begin{array}{l l  l}
\observefc{\faila{\tau}}{\Pi}{p} & = &  \circ \\
\observefc{\select{e_1}{e_2}}{\delegcontext}{p} & = &
      \select{\observefc{e_1}{\Pi}{p}}{\observefc{e_2}{\Pi}{p}} \\
\observefc{\compare{e_1}{e_2}}{\delegcontext}{p} & = &
\compare{\observefc{e_1}{\Pi}{p}}{\observefc{e_2}{\Pi}{p}} \\ 
\observefc{\distcon{e}{c}{s}}{\Pi}{\ell} & = &
      \begin{cases}
        \observefc{e}{\Pi}{\ell} & s=\emptystack \\
        \observefc{e}{\Pi}{\ell}\& \observefc{s}{\Pi}{\ell} \\
      \end{cases} \\[1.25em] 
\observefc{\stackapp{e}{c}{s}}{\Pi}{\ell} & = &
     \begin{cases}
       \observefc{e}{\Pi}{\ell}  & s=\emptystack \\
       \observefc{e}{\Pi}{\ell} :: \observefc{s}{\Pi}{\ell} \\
     \end{cases}\\ [1.25em]
\observefc{E[\runa{\tau}{e}{c}]}{\Pi}{\ell} & = & \observefc{E[e]}{\Pi}{\ell}\\
\observefc{\ret{e}{c}}{\Pi}{\ell} & = & \observefc{e}{\Pi}{\ell} \\ 
\end{array}
\]
}
\caption[Observation function for intermediate FLAQR terms]{Observation function for intermediate FLAQR terms (extended from FLAC \cite{jflac}).}
\label{fig:observe}
\end{figure*}
